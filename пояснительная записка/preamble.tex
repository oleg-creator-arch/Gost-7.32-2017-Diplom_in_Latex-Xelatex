%%% Преамбула %%%

\usepackage{fontspec} % XeTeX
\usepackage{xunicode} % Unicode для XeTeX
\usepackage{xltxtra}  % Верхние и нижние индексы
\usepackage{pdfpages} % Вставка PDF
\usepackage{comment}

\begin{comment}
\usepackage[% 
backend=biber, %подключение пакета biber (тоже нужен)
bibstyle=gost-numeric, %подключение одного из четырех главных стилей biblatex-gost 
citestyle=numeric-comp, %подключение стиля стиля (а вот!) 
language=auto, %указание сортировки языков
babel=other, %указание языков
sorting=ntvy, %тип сортировки в библиографии
doi=false, 
eprint=false, 
isbn=false, 
dashed=false, 
url=false %все false выключают отображение полей, заполненных в библиографической базе, но не актуальных для печатного листа
]{biblatex}
\DeclareFieldFormat{postnote}{#1} %убирает с. и p. 
    \map{
    	\renewcommand*{\multicitedelim}{\addsemicolon\space} % добавляет точку с запятой и пробел (; ) в перечислении
    	\renewcommand*{\postnotedelim}{\addcolon\space} % добавляет двоеточие и пробел (: ) перед номером страницы
    	\DeclareSourcemap{
    		\maps[datatype=bibtex]{
    			\map{
    				\step[fieldsource=langid, match=russian, final]
    				\step[fieldset=presort, fieldvalue={a}]
    			}
      \step[fieldsource=langid, notmatch=russian, final]
      \step[fieldset=presort, fieldvalue={z}]
    }
  }
}

\usepackage{polyglossia}
\setdefaultlanguage{russian} %язык по умолчанию
\setotherlanguage{english}
%\addbibresource{bib.bib}
 %\bibliographystyle{unsrt}
%\usepackage{scrextend} % размер 14pt
\end{comment}





\usepackage{listings} % Оформление исходного кода
\lstset{
    %basicstyle=\small\ttfamily, % Размер и тип шрифта
    %breaklines=true,            % Перенос строк
    %tabsize=2,                  % Размер табуляции
    %frame=single,               % Рамка
    %literate={--}{{-{}-}}2,     % Корректно отображать двойной дефис
    %literate={---}{{-{}-{}-}}3  % Корректно отображать тройной дефис
    basicstyle=\footnotesize\ttfamily,
	language=[Sharp]C, % Или другой ваш язык -- см. документацию пакета
	commentstyle=\color{comment},
	numbers=left,
	numberstyle=\tiny\color{plain},
	numbersep=5pt,
	tabsize=4,
	extendedchars=\true,
	breaklines=true,
	keywordstyle=\color{blue},
	frame=b,
	stringstyle=\ttfamily\color{string}\ttfamily,
	showspaces=false,
	showtabs=false,
	xleftmargin=17pt,
	framexleftmargin=17pt,
	framexrightmargin=5pt,
	framexbottommargin=4pt,
	showstringspaces=false,
	inputencoding=utf8x,
	keepspaces=true
}

% Шрифты, xelatex
\defaultfontfeatures{Ligatures=TeX}
\setmainfont[Scale=.976]{Times New Roman} % Нормоконтроллеры хотят именно его
\newfontfamily\cyrillicfont{Times New Roman}
% \setsansfont{Liberation Sans} % Тут я его не использую, но если пригодится
\setmonofont{FreeMono} % Моноширинный шрифт для оформления кода

% Формулы
\usepackage{mathtools,unicode-math} % Не совместим с amsmath
\setmathfont{XITS Math}             % Шрифт для формул: https://github.com/khaledhosny/xits-math
\numberwithin{equation}{section}    % Формула вида секция.номер

% Русский язык
\usepackage{polyglossia}
\setdefaultlanguage{russian}%язык по умолчанию
\setotherlanguage{english}
% Абзацы и списки
\usepackage{enumerate}   % Тонкая настройка списков
\usepackage{indentfirst} % Красная строка после заголовка
\usepackage{float}       % Расширенное управление плавающими объектами
\usepackage{multirow}    % Сложные таблицы

% Пути к каталогам с изображениями
\usepackage{graphicx} % Вставка картинок и дополнений
\graphicspath{{images/}}

% Формат подрисуночных записей
\usepackage{chngcntr}

% Сбрасываем счетчик таблиц и рисунков в каждой новой главе
\counterwithin{figure}{section}
\counterwithin{table}{section}

% Гиперссылки
\usepackage{hyperref}
\hypersetup{
    colorlinks, urlcolor={black}, % Все ссылки черного цвета, кликабельные
    linkcolor={black}, citecolor={black}, filecolor={black},
}

% Оформление библиографии и подрисуночных записей через точку
\makeatletter
%\renewcommand*{\@biblabel}[1]{\hfill#1.}
\renewcommand*\l@section{\@dottedtocline{1}{1em}{1em}}
\renewcommand{\thefigure}{\thesection.\arabic{figure}} % Формат рисунка секция.номер
\renewcommand{\thetable}{\thesection.\arabic{table}}   % Формат таблицы секция.номер
\def\redeflsection{\def\l@section{\@dottedtocline{1}{0em}{10em}}}
\makeatother

\renewcommand{\baselinestretch}{1.5} % Полуторный межстрочный интервал
\parindent 1.5cm % Абзацный отступ

%\sloppy             % Избавляемся от переполнений
%\hyphenpenalty=1000 % Частота переносов
\clubpenalty=10000  % Запрещаем разрыв страницы после первой строки абзаца
\widowpenalty=10000 % Запрещаем разрыв страницы после последней строки абзаца

% Отступы у страниц
\usepackage{geometry}
\geometry{left=3cm}
\geometry{right=1.5cm}
\geometry{top=2cm}
\geometry{bottom=2cm}

% Списки
\usepackage{enumitem}
\setlist[enumerate,itemize]{leftmargin=12.7mm} % Отступы в списках

\makeatletter
    \AddEnumerateCounter{\asbuk}{\@asbuk}{м)}
\makeatother
\setlist{nolistsep}                           % Нет отступов между пунктами списка
\renewcommand{\labelitemi}{--}                % Маркер списка --
\renewcommand{\labelenumi}{\asbuk{enumi})}    % Список второго уровня
\renewcommand{\labelenumii}{\arabic{enumii})} % Список третьего уровня

% Содержание
\usepackage{tocloft}
\renewcommand{\cfttoctitlefont}{\hspace{0.38\textwidth}\MakeTextUppercase} % СОДЕРЖАНИЕ
\renewcommand{\cftsecfont}{\hspace{0pt}}            % Имена секций в содержании не жирным шрифтом
\renewcommand\cftsecleader{\cftdotfill{\cftdotsep}} % Точки для секций в содержании
\renewcommand\cftsecpagefont{\mdseries}             % Номера страниц не жирные
\setcounter{tocdepth}{3}                            % Глубина оглавления, до subsubsection

% Список иллюстративного материала
\renewcommand{\cftloftitlefont}{\hspace{0.17\textwidth}\MakeTextUppercase}
\renewcommand{\cftfigfont}{Рисунок }
\addto\captionsrussian{\renewcommand\listfigurename{Список иллюстративного материала}}

% Список табличного материала
\renewcommand{\cftlottitlefont}{\hspace{0.2\textwidth}\MakeTextUppercase}
\renewcommand{\cfttabfont}{Таблица }
\addto\captionsrussian{\renewcommand\listtablename{Список табличного материала}}

% Нумерация страниц посередине снизу
\usepackage{fancyhdr}
\pagestyle{fancy}
\fancyhf{}
\cfoot{\textrm{\thepage}}
\fancyheadoffset{0mm}
\fancyfootoffset{0mm}
\setlength{\headheight}{17pt}
\renewcommand{\headrulewidth}{0pt}
\renewcommand{\footrulewidth}{0pt}
\fancypagestyle{plain}{
    \fancyhf{}
    \cfoot{\textrm{\thepage}}
}

% Формат подрисуночных надписей
\RequirePackage{caption}
\DeclareCaptionLabelSeparator{defffis}{ -- } % Разделитель
\captionsetup[figure]{justification=centering, labelsep=defffis, format=plain} % Подпись рисунка по центру
\captionsetup[table]{justification=raggedright, labelsep=defffis, format=plain, singlelinecheck=false} % Подпись таблицы слева
\addto\captionsrussian{\renewcommand{\figurename}{Рисунок}} % Имя фигуры

% Пользовательские функции
\newcommand{\addimg}[4]{ % Добавление одного рисунка
    \begin{figure}
        \centering
        \includegraphics[width=#2\linewidth]{#1}
        \caption{#3} \label{#4}
    \end{figure}
}
\newcommand{\addimghere}[4]{ % Добавить рисунок непосредственно в это место
    \begin{figure}[H]
        \centering
        \includegraphics[width=#2\linewidth]{#1}
        \caption{#3} \label{#4}
    \end{figure}
}
\newcommand{\addtwoimghere}[5]{ % Вставка двух рисунков
    \begin{figure}[H]
        \centering
        \includegraphics[width=#2\linewidth]{#1}
        \hfill
        \includegraphics[width=#3\linewidth]{#2}
        \caption{#4} \label{#5}
    \end{figure}
}

% Заголовки секций в оглавлении в верхнем регистре
\usepackage{textcase}
\makeatletter
\let\oldcontentsline\contentsline
\def\contentsline#1#2{
    \expandafter\ifx\csname l@#1\endcsname\l@section
        \expandafter\@firstoftwo
    \else
        \expandafter\@secondoftwo
    \fi
    {\oldcontentsline{#1}{\MakeTextUppercase{#2}}}
    {\oldcontentsline{#1}{#2}}
}
\makeatother

% Оформление заголовков
\usepackage[compact,explicit]{titlesec}
\titleformat{\section}{\Large}{}{12.5mm}{\centering{\thesection\quad\MakeTextUppercase{#1}}\vspace{1.5em}}
\titleformat{\subsection}[block]{\vspace{1em}\Large}{}{12.5mm}{\thesubsection\quad#1\vspace{1em}}
\titleformat{\subsubsection}[block]{\vspace{1em}\normalsize}{}{12.5mm}{\thesubsubsection\quad#1\vspace{1em}}
\titleformat{\paragraph}[block]{\normalsize}{}{12.5mm}{\MakeTextUppercase{#1}}

% Секции без номеров (введение, заключение...), вместо section*{}
\newcommand{\anonsection}[1]{
    \phantomsection % Корректный переход по ссылкам в содержании
    \paragraph{\Large\centerline{{#1}}\vspace{1em}}
    \addcontentsline{toc}{section}{#1}
}

% Секция для не включеных в содержание
\newcommand{\ancontent}[1]{
    \paragraph{\Large\centerline{{#1}}\vspace{1em}}
}

% Секция для списка иллюстративного материала
\newcommand{\lof}{
    \phantomsection{\Large}
    \listoffigures
    \addcontentsline{toc}{section}{\listfigurename}
}

% Секция для списка табличного материала
\newcommand{\lot}{
    \phantomsection{\Large}
    \listoftables
    \addcontentsline{toc}{section}{\listtablename}
}

% Секции для приложений
\newcommand{\appsection}[1]{
    \phantomsection
    \paragraph{\Large\centerline{{#1}}}
    \addcontentsline{toc}{section}{{#1}}
}

\begin{comment}
% Библиография: отступы и межстрочный интервал
\makeatletter
\renewenvironment{thebibliography}[1]
    {%\section*{СПИСОК ИСПОЛЬЗОВАННЫХ ИСТОЧНИКОВ}
        \list{\@biblabel{\@arabic\c@enumiv}}
           {\settowidth\labelwidth{\@biblabel{#1}}
            \leftmargin\labelsep
            \itemindent 16.7mm
            \@openbib@code
            \usecounter{enumiv}
            \let\p@enumiv\@empty
            \renewcommand\theenumiv{\@arabic\c@enumiv}
        }
        \setlength{\itemsep}{0pt}
    }
\makeatother
\end{comment}
\usepackage{lastpage} % Подсчет количества страниц
\setcounter{page}{3}  % Начало нумерации страниц

\begin{comment}


\usepackage[% 
backend=biber, %подключение пакета biber (тоже нужен)
bibstyle=gost-numeric, %подключение одного из четырех главных стилей biblatex-gost 
citestyle=numeric-comp, %подключение стиля стиля (а вот!) 
language=auto, %указание сортировки языков
babel=other, %указание языков
sorting=ntvy, %тип сортировки в библиографии
doi=false, 
eprint=false, 
isbn=false, 
dashed=false, 
url=false %все false выключают отображение полей, заполненных в библиографической базе, но не актуальных для печатного листа
]{biblatex}

\usepackage[style=gost,sorting=none]{biblatex}
\addbibresource {simple_utf8.bib}






\usepackage[% 
backend=biber, %подключение пакета biber (тоже нужен)
bibstyle=gost-numeric, %подключение одного из четырех главных стилей biblatex-gost 
citestyle=numeric-comp, %подключение стиля стиля (а вот!) 
language=auto, %указание сортировки языков
babel=other, %указание языков
sorting=ntvy, %тип сортировки в библиографии
doi=false, 
eprint=false, 
isbn=false, 
dashed=false, 
url=false %все false выключают отображение полей, заполненных в библиографической базе, но не актуальных для печатного листа
]{biblatex}
\DeclareFieldFormat{postnote}{#1} %убирает с. и p. 
\renewcommand*{\multicitedelim}{\addsemicolon\space} % добавляет точку с запятой и пробел (; ) в перечислении
\renewcommand*{\postnotedelim}{\addcolon\space} % добавляет двоеточие и пробел (: ) перед номером страницы
\DeclareSourcemap{
	\maps[datatype=bibtex]{
		\map{
			\step[fieldsource=langid, match=russian, final]
			\step[fieldset=presort, fieldvalue={a}]
		}
		\map{
			\step[fieldsource=langid, notmatch=russian, final]
			\step[fieldset=presort, fieldvalue={z}]
		}
	}
}
\addbibresource{bib.bib}
\end{comment}

\bibliographystyle{gost-numeric.bbx}
\usepackage[parentracker=true,
backend=biber,
hyperref=false,
bibencoding=utf8,
style=gost-numeric,
language=auto,
autolang=other,
citestyle=gost-numeric,
defernumbers=true,
bibstyle=gost-numeric,
sorting=ntvy,
]{biblatex}
\addbibresource{bib.bib}

