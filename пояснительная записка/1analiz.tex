\section{Описание и анализ предметной области}
\begin{comment}

пример рисунка
\addimghere{photo.jpg}{0.25}{1}{1}
\subsection{пример второго уровня заголовка}
@@ -14,10 +16,87 @@ \subsection{пример второго уровня заголовка}
\end{tabular}
\end{center}
\end{table}
\addimghere{commpa}{1}{Организация структуры на примере диаграммы компонентов}{commpa}
\addimghere{communication}{1}{Диаграмма коммуникации компонентов}{commu}
\addimghere{graph}{0.5}{Граф показывающий связь между объектами и передаваемый объем информации}{graph}
\section{Математическая модель}
бла бла 
\end{comment}

\subsection{Организационная структура САО РАН}
Цель деятельности САО РАН осуществлять научные исследования за объектами вселенной с земли, издание научных журналов и научных трудов, в которых публикуются результаты исследований ученых.

Для реализации своих целей у обсерватории существуют такие инструменты, как оптический телескоп БТА (Большой Телескоп Азимутальный) с диаметром главного зеркала 6 м и радиотелескоп РАТАН-600 (РадиоТелескоп Академии Наук) с кольцевой многоэлементной антенной диаметром 600 м. Телескопы (инструменты) открытого коллективного пользования, и допускают широкую интеграцию с астрономическими сообществами по всему миру. Время наблюдений распределяется национальным комитетом по тематике российских телескопов сокращенно НКТРТ.

По информации с сайта САО РАН: <<Обсерватория расположена в Зеленчукском районе Карачаево-Черкесской Республики Российской Федерации. БТА установлен на склонах г. Пастухова на высоте 2100 метров над уровнем моря. Здесь же находятся два малых телескопа диаметром 1 и 0.6 метров. РАТАН-600 сооружен в 20 км от БТА на окраине станицы Зеленчукской на высоте 970 метров. Научный поселок Нижний Архыз (лабораторные и служебные корпуса САО и жилые дома сотрудников) построены на берегу реки Большой Зеленчук. Обсерватория имеет филиал в Санкт-Петербурге (Пулково). В САО работает 430 сотрудников, из которых чуть более 100 - научные работники. САО РАН по Распоряжению Правительства РФ от 27.06.2018 1293-р относится к организациям, подведомственным Министерству науки и высшего образования Российской Федерации (Минобрнауки России)>>\cite{Korn}. Из чего следует, что объекты находятся на расстоянии друг от друга, и через переопределенные связи осуществляют коммуникацию. Внешняя, неформальная, межличностная и организационная коммуникация не рассматриваться. Внутренняя коммуникация (обмен информацией между объектами, внутри САО РАН), происходит средством сети Интернет. 

 

САО РАН имеет иерархический тип структуры управления с линейно-штабной организационной структурой. Условно можно выделить 5 объектов, представим их  на рисунке \ref{pic:org}
\addimghere{resources/org}{1}{Условная организационная структура САО РАН}{pic:org}
%описать словесно условную структуру



\subsection{Описание процессов}
\addimghere{resources/commpa}{1}{Диаграмма компонентов}{pic:commpa}
\addimghere{resources/communication}{1}{Диаграмма коммуникации компонентов}{pic:commu}

\subsection{Формирование математической модели проблемы}
Воспользуемся математическим моделирование для выявления <<узких мест>> в функционировании САО РАН, и нужных параметров, при оптимизации которых улучшится результат деятельности

Опишем информационный обмен между подсистемами в системе через математическую модель. Ограничимся рамками (будем предполагать, что система замкнута), информационный обмен будет  происходит в пределах пяти подсистем, таких как <<администрация>>, <<лаборатория информатики>>, <<радиоастрономический сектор>>, <<вспомогательное-техническое подразделение>>, <<оптический сектор>>, и обозначим их соответственно $a, b, c, d, e$. Подсистемы обладают запасами информационных ресурсов, допустим эти запасы изменяются за счет обмена информацией между подсистемами. Для наглядности покажем граф связи между объектами, и объем информационного оборота между звеньями за ед. времени ТБ в день (рисунок \ref{pic:graph}).

\addimghere{resources/graph}{0.5}{Граф показывающий связь между объектами и передаваемый объем информации}{pic:graph}



Нужно понимать, что информационный ресурс не переходит из одного объекта к другому, а накапливается и распределяется по системе, и общее количество семантической информации в системе остается неизменным. К примеру, если объект $A$ передаст (поделиться) информацией с другим каким-либо объектом, информация у объекта $A$ не уменьшиться, а у другого увеличиться.

Рассмотрим пару объектов $A$ и $B$, которые обмениваться информацией, пусть объект $A$ передает информацию, а объект $B$ её принимает.

Информация которая иметься только у объекта, то есть  $i \left( i \in \left\{A,B \right\} \right)$ обозначим как $K_i$. Количество переданной информации другому объекту обозначим $J_A$. Информация, которая была воспринята другой подсистемой $B$ обозначим как $I$, а которая была не воспринята $S_A$. Для объекта $A$ введем функцию результативности решения это будет $P_A=P_A(K_A,J_A)$ данная функция показывает от чего зависит результативность решения у объекта $A$ 	это количество информации, которая она имеет, и количество переданной информации. Для объекта $B$ результативность будет зависеть от количества информации, которое в ней иметься, и принятой информации $P_B=P_B(K_B,I)$. 
Можем сократить аргументы, в уравнениях состояний, потому что система замкнута значит количество информации $J_A+K_A$ и $K_B$ не изменяться во времени. Получаем итоговые формулы уравнений состояний:
\begin{equation}\label{eq:sost1}
P_A=P_A(J_A),
\end{equation}
\begin{equation}\label{eq:sost2}
P_B=P_B(I). 
\end{equation}

За счет обмена информацией между объектами, происходит изменение величины $P_i = \left( i \in \left\{A,B \right\} \right)$, обозначим формулой ценность информации :
\begin{equation}\label{eq:cenoostinfo1}
\upsilon_A = \frac{dP_A}{dJ_A},
\end{equation}
\begin{equation}\label{eq:cenoostinfo2}
\upsilon_B = \frac{dP_B	}{dI}.
\end{equation}

Значение $\upsilon_i$ при $i \in \left\{A,B \right\}$ показывает как хорошо подсистемы взаимодействуют между собой. При $\upsilon_i > 0$ $i$-подсистема может положительно обмениваться информацией, а при $\upsilon_i < 0$ $i$-подсистема препятствует передачи или приему информации. Представим введи формулы интенсивность потока информации:
\begin{equation}\label{eq:intensive}
q_a(\upsilon_A,\upsilon_B) = \alpha (\upsilon_A + \upsilon_B).
\end{equation}
Где $\alpha$ -- это размерный коэффициент пропорциональности.

Составим формулу интенсивности в пределах времени:
\begin{equation}\label{eq:intensiveA}
q_a(\upsilon_A,\upsilon_B) = \frac{dJ_A}{dt}
\end{equation}

Интенсивность информации принимаемого объекта или подсистемы, всегда меньше интенсивности отправителя, так как есть некоторая информация, которая была не воспринята получателем. Обозначим коэффициент (параметр) воспринятой информации принимаемым объектом (подсистемой) как $p(q_A)$. Обозначим формулу интенсивности для получателя:
\begin{equation}\label{eq:intensiveB}
q_B = p(q_A)q_A
\end{equation}
Можем заметить что при высокой интенсивности $q_A \rightarrow \infty$ коэффициент $p(q_A)$ стремиться к нулю, и информация меньше воспринимается, а при $q_A \rightarrow 0$ почти вся преданная информация воспринимается, обозначим это формулами:
\begin{equation}\label{eq:lim1}
\lim_{q_A \rightarrow 0} p(q_A) = 1,
\end{equation}
\begin{equation}\label{eq:lim0}
\lim_{q_A \rightarrow \infty} p(q_A) = 0.
\end{equation}

Таким образом построив математическую модель мы нашли <<узкое место>>, что в системе высокая интенсивность обмена информацией между подсистемами, и что некоторая информация может быть не воспринята подсистемой. Следовательно проблема заключается в большой интенсивности информационного потока между подсистемами.

\subsection{Полная постановка задачи ВКР}
Проведя анализ предметной области, рассмотрели основные выполняемые процессы в системе, схему организационной структуры, и в результате построения математической модели была выявлена проблема, что информационной оборот в системе очень интенсивный. Следовательно нашей целью является снизить интенсивность информации, и оптимизировать коммуникацию между подсистемами.

Для решения постановленной цели выделим ряд задач:
\begin{itemize}
	\item Провести оптимизацию модели и процессов,
	\item Построить процессы по оптимизированной модели,
	\item Сформировать параметры и требования к разработке,
	\item Сравнить аналоги удовлетворяющие требованиям,
	\item Спроектировать разработку,
	\item Реализовать разработку.
\end{itemize}
\pagebreak