\section{Описание и анализ предметной области}
пример рисунка
\addimghere{photo.jpg}{0.25}{1}{1}
\subsection{пример второго уровня заголовка}
пример таблицы 
\begin{table}[H]
\caption{Исправьте это на подпись к таблице}
\label{tabular:timesandtenses}
\begin{center}
\begin{tabular}{ccc}
Расширение краёв: & \textbf{1,0-1,4} & размер ФРТ \\
Аподизация: & \textbf{0,25-0,30} & размер ФРТ \\
Сглаживания краёв: & \textbf{0,25-0,50}& размер ФРТ \\
\end{tabular}
\end{center}
\end{table}
\addimghere{resources/commpa}{1}{Организация структуры на примере диаграммы компонентов}{pic:commpa}
\addimghere{resources/communication}{1}{Диаграмма коммуникации компонентов}{pic:commu}
\addimghere{resources/graph}{0.5}{Граф показывающий связь между объектами и передаваемый объем информации}{pic:graph}
\section{Математическая модель}
бла бла 

\pagebreak