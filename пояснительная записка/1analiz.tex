\section{Описание и анализ предметной области}
\begin{comment}

пример рисунка
\addimghere{photo.jpg}{0.25}{1}{1}
\subsection{пример второго уровня заголовка}
пример таблицы 
\begin{table}[H]
\caption{Исправьте это на подпись к таблице}
\label{tabular:timesandtenses}
\begin{center}
\begin{tabular}{ccc}
Расширение краёв: & \textbf{1,0-1,4} & размер ФРТ \\
Аподизация: & \textbf{0,25-0,30} & размер ФРТ \\
Сглаживания краёв: & \textbf{0,25-0,50}& размер ФРТ \\
\end{tabular}
\end{center}
\end{table}
\end{comment}

\subsection{Организационная структура САО РАН}
САО РАН имеет иерархический тип структуры управления с линейно-штабной организационной структурой. Условно можно выделить 5 объектов, представим их  на рисунке \ref{pic:org}
\addimghere{resources/org}{1}{Условная организационная структура САО РАН}{pic:org}
%описать словесно условную структуру
\subsection{Описание процессов}
\addimghere{resources/commpa}{1}{Диаграмма компонентов}{pic:commpa}
\addimghere{resources/communication}{1}{Диаграмма коммуникации компонентов}{pic:commu}
\addimghere{resources/graph}{0.5}{Граф показывающий связь между объектами и передаваемый объем информации}{pic:graph}
\subsection{Формирование математической модели чего-то ...}
Воспользуемся математическим моделирование для выявления <<узких мест>> в функционировании САО РАН, и нужных параметров, при оптимизации которых улучшится результат деятельности

Опишем информационный обмен между подсистемами в системе через математическую модель. Ограничимся рамками (будем предполагать, что система замкнута), информационный обмен будет  происходит в пределах пяти подсистем, таких как <<администрация>>, <<лаборатория информатики>>, <<радиоастрономический сектор>>, <<вспомогательное-техническое подразделение>>, <<оптический сектор>>, и обозначим их соответственно $a, b, c, d, e$. Подсистемы обладают запасами информационных ресурсов, допустим эти запасы изменяются за счет обмена информацией между подсистемами. 

Нужно понимать, что информационный ресурс не переходит из одного объекта к другому, а накапливается и распределяется по системе, и общее количество семантической информации в системе остается неизменным. К примеру, если объект $A$ передаст (поделиться) информацией с другим каким-либо объектом, информация у объекта $A$ не уменьшиться, а у другого увеличиться.

Рассмотрим пару объектов $A$ и $B$, которые обмениваться информацией, пусть объект $A$ передает информацию, а объект $B$ её принимает. Информация которая иметься только у объекта i={A,B}  объем информации в Они оба заинтересованы в решении определенной задачи, введем функцию P   

 


\pagebreak