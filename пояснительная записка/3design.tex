\section{Проектирование информационной системы}

\subsection{Формирование параметров и требований к ИС}
%Представим перечни, которыми должны обладать разрабатываемая система.
Система должна выполнять перечень задач:
\begin{itemize}
	\item создание аккаунта любому пользователю, который так же может быть вне системы САО РАН, 
	\item создание блогов,
	\item создание и редактирование постов(оформление статей в блогах),
	\item приглашать других пользователей в качестве авторов,
	\item видеть список постов внутри блога,
	\item иметь возможность сохранить пост, чтобы позже его опубликовать. 
\end{itemize}

Из перечня задач сформируем требования к разрабатываемой системе:
\begin{itemize}
	\item создание и видение личных блогов,
	\item создание блогов командой,
	\item введение блогов командой,
	\item наличие удобного редактора для текста,
	\item сохранить пост, без его публикации. 
\end{itemize}

\subsection{Сравнение аналогов}
В сети Интернет уже давно развиваются разные системы с блогами, рассмотрим их.
 
Наиболее популярными системами являются .


\pagebreak