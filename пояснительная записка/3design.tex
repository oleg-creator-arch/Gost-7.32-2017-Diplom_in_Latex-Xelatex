\section{Проектирование информационной системы}

\subsection{Формирование параметров и требований к ИС}
%Представим перечни, которыми должны обладать разрабатываемая система.
Система должна выполнять перечень задач:
\begin{itemize}
	\item создание аккаунта любому пользователю, который так же может быть вне системы САО РАН, 
	\item создание блогов,
	\item создание и редактирование постов(оформление статей в блогах),
	\item приглашать других пользователей в качестве авторов,
	\item видеть список постов внутри блога,
	\item иметь возможность сохранить пост, чтобы позже его опубликовать. 
\end{itemize}

Из перечня задач сформируем требования к разрабатываемой системе:
\begin{itemize}
	\item создание и ведение личных(закрытых) блогов,
	\item создание блогов командой,
	\item ведение блогов командой,
	\item наличие удобного редактора для текста,
	\item сохранить пост, без его публикации. 
\end{itemize}

\subsection{Сравнение аналогов}
В сети Интернет уже давно развиваются разные системы с блогами, наиболее популярными системами являются Medium, Хабр и Blogger, рассмотрим их более подробно.

\pagebreak
\subsubsection{Medium}
Medium платформа для социальной журналистики, позволяет создавать аккаунты, писать стати так называемые <<Story>>, коллективные блоги называемые <<Publications>>, покажем на рисунке \ref{pic:medium} как выглядит главная страница платформы.

\addimghere{resources/medium}{0.9}{Главная страница Medium}{pic:medium}

Присутствует текстовый редактор, показан на рисунке \ref{pic:textmedium}, среднего качества. Нужно оформлять платную подписку, чтобы получить более полный функционал платформы.

\addimghere{resources/textmedium}{0.9}{Текстовый редактор платформы Medium}{pic:textmedium}

\subsubsection{Хабр}

Хабр платформа с тематическими командными блогами, где каждый может написать свою статью, покажем на рисунке \ref{pic:habr} как выглядит главная страница платформы.

\addimghere{resources/habr}{0.9}{Главная страница Хабр}{pic:habr}

Платформа позволяет зарегистрироваться и писать статьи по своей тематике, так называемые <<Хабы>>. От новых пользователей стати проходят модерацию. После того как пользователь покажет себя с хорошей стороны, ему станет доступен полный доступ, и новые статьи от него будут публиковаться сразу.

\pagebreak

Присутствует текстовый редактор, показан на рисунке \ref{pic:texthabr}, с возможностью форматирования специальный разметкой <<Markdown>>, что делает его понятным и удобным для пользователя.

\addimghere{resources/texthabr}{0.9}{Текстовый редактор платформы Хабр}{pic:texthabr}

Пользователь не может вести отдельный личный блог, без попадания его статьи в общие блоги. Ограничен список тематик на которые можно писать статьи из-за модерации, так как площадка позиционирует себя, по большей части, как техно-научная.

\pagebreak

\subsubsection{Blogger}

Blogger платформа специально предназначенная для ведения блогов, покажем на рисунке \ref{pic:blogger} как выглядит главная страница платформы.

\addimghere{resources/blogger}{1}{Главная страница платформы blogger}{pic:blogger}

Платформа позволяет быстро авторизоваться через Google-аккаунт и создавать свои блоги.

Присутствует текстовый редактор, показан на рисунке \ref{pic:texblogger}, понятный и удобный для пользователя с возможностью поиска совпадений.

\addimghere{resources/texblogger}{1}{Текстовый редактор платформы blogger}{pic:texblogger}

У платформы отсутствует возможность вести командные блоги.

Покажем результаты сравнения аналогов в таблице \ref{tabular:analogi}.

\begin{table}[H]
	\caption{Результат сравнения аналогов}
	\label{tabular:analogi}
	\begin{center}
		\begin{tabular}{|l|l|l|l|}
			\hline
			\multicolumn{1}{|c|}{Критерий}             & \multicolumn{1}{c|}{Medium} & \multicolumn{1}{c|}{Хабр} & \multicolumn{1}{c|}{Blogger} \\ \hline
			Создание и ведение личных(закрытых) блогов & +                           & -                         & +                            \\ \hline
			Создание блогов командой                   & +                           & -                         & -                            \\ \hline
			Ведение блогов командой                    & +                           & +                         & -                            \\ \hline
			Наличие удобного редактора для текста      & -                           & +                         & +                            \\ \hline
			Сохранить пост, без его публикации         & +                           & -                         & +                            \\ \hline
		\end{tabular}
	\end{center}
\end{table}
В результате анализа аналогов получили, что не одно из готовых платформ не подходит для наших целей следовательно нужно спроектировать и разработать информационную систему.

\pagebreak

\subsection{Проектирование информационной системы}

Покажем на рисунке \ref{pic:precedent} Use Case диаграмму прецедентов информационной системы, которая отобразить множество выполняемых системой функций и взаимодействия с ней пользователя. 
 
\addimghere{resources/precedent}{1}{Use Case диаграмма прецедентов ИС}{pic:precedent}

Из диаграммы видны действия, которые может выполнять пользователь в системе.

actor:
\begin{itemize}
	\item регистрация, чтобы получить расширенный функционал 
	\item просмотреть блог по ссылке, чтобы по полученной ссылке смотреть посты, для этого действие расширяется предоставлением каталогов постов в блоге.
\end{itemize}

\pagebreak

SuperActor потомок от actor:
\begin{itemize}   
	\item авторизоваться, переход к расширенному функционалу,
	\item наличие удобного редактора для текста,
	\item сохранить пост, без его публикации. 
\end{itemize}





\pagebreak