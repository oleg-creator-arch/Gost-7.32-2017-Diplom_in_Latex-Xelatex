\section{Проектирование информационной системы}

\subsection{Формирование параметров и требований к ИС}
%Представим перечни, которыми должны обладать разрабатываемая система.
Система должна выполнять перечень задач:
\begin{itemize}
	\item создание аккаунта любому пользователю, который так же может быть вне системы САО РАН, 
	\item создание блогов,
	\item создание и редактирование постов(оформление статей в блогах),
	\item приглашать других пользователей в качестве авторов,
	\item видеть список постов внутри блога,
	\item иметь возможность сохранить пост, чтобы позже его опубликовать. 
\end{itemize}

Из перечня задач сформируем требования к разрабатываемой системе:
\begin{itemize}
	\item создание и ведение личных(закрытых) блогов,
	\item создание блогов командой,
	\item ведение блогов командой,
	\item наличие удобного редактора для текста,
	\item сохранить пост, без его публикации. 
\end{itemize}

\pagebreak
\subsection{Сравнение аналогов}
Для сравнения аналогов воспользуемся методом SMART, для этого введем значения критериям в табл. \ref{tabular:samrt}

\begin{table}[H]
	\caption{оценка критериев} 
	\label{tabular:samrt}
	\begin{center}
\begin{tabular}{|l|l|}
\hline
\multicolumn{1}{|c|}{Критерии}             & \multicolumn{1}{c|}{Баллы} \\ \hline
Создание и ведение личных(закрытых) блогов & 100                        \\ \hline
Создание блогов командой                   & 85                         \\ \hline
Ведение блогов командой                    & 75                         \\ \hline
Наличие удобного редактора для текста      & 65                         \\ \hline
Сохранить пост, без его публикации         & 55                         \\ \hline
\end{tabular}
	\end{center}
\end{table}

Произведем нормировку критерием по формуле \ref{eq:smart}

\begin{equation}\label{eq:smart}
W_i  = \frac{{A_i }}{{\sum\nolimits_i^n {A_i } }}
\end{equation}

Где $A_i$ -- баллы критерия,

$n$ -- количество критериев.

Приведем результат нормировки в табл. \ref{tabular:normirov}

\begin{table}[H]
	\caption{оценка критериев} 
	\label{tabular:normirov}
	\begin{center}
\begin{tabular}{|l|l|l|}
\hline
\multicolumn{1}{|c|}{Критерии}             & \multicolumn{1}{c|}{Баллы} & Нормировка \\ \hline
Создание и ведение личных(закрытых) блогов & 100                        & 0,26       \\ \hline
Создание блогов командой                   & 85                         & 0,22       \\ \hline
Ведение блогов командой                    & 75                         & 0,2        \\ \hline
Наличие удобного редактора для текста      & 65                         & 0,17       \\ \hline
Сохранить пост, без его публикации         & 55                         & 0,14       \\ \hline
Сумма                                      & 380                        & 1          \\ \hline
\end{tabular}
	\end{center}
\end{table}

Приведем краткую информацию по альтернативам. В сети Интернет уже давно развиваются разные системы с блогами, наиболее популярными системами являются Medium, Хабр и Blogger, рассмотрим их более подробно.

\pagebreak
\subsubsection{Medium}
Medium платформа для социальной журналистики, позволяет создавать аккаунты, писать стати так называемые <<Story>>, коллективные блоги называемые <<Publications>>, покажем на рис. \ref{pic:medium} как выглядит главная страница платформы.

\addimghere{resources/medium}{0.9}{Главная страница Medium}{pic:medium}

Присутствует текстовый редактор, показан на рис. \ref{pic:textmedium}, среднего качества. Нужно оформлять платную подписку, чтобы получить более полный функционал платформы.

\addimghere{resources/textmedium}{0.9}{Текстовый редактор платформы Medium}{pic:textmedium}

\subsubsection{Хабр}

Хабр платформа с тематическими командными блогами, где каждый может написать свою статью, покажем на рис. \ref{pic:habr} как выглядит главная страница платформы.

\addimghere{resources/habr}{0.9}{Главная страница Хабр}{pic:habr}

Платформа позволяет зарегистрироваться и писать статьи по своей тематике, так называемые <<Хабы>>. От новых пользователей стати проходят модерацию. После того как пользователь покажет себя с хорошей стороны, ему станет доступен полный доступ, и новые статьи от него будут публиковаться сразу.

\pagebreak

Присутствует текстовый редактор, показан на рис. \ref{pic:texthabr}, с возможностью форматирования специальный разметкой <<Markdown>>, что делает его понятным и удобным для пользователя.

\addimghere{resources/texthabr}{0.9}{Текстовый редактор платформы Хабр}{pic:texthabr}

Пользователь не может вести отдельный личный блог, без попадания его статьи в общие блоги. Ограничен список тематик на которые можно писать статьи из-за модерации, так как площадка позиционирует себя, по большей части, как техно-научная.

\pagebreak

\subsubsection{Blogger}

Blogger платформа специально предназначенная для ведения блогов, покажем на рис. \ref{pic:blogger} как выглядит главная страница платформы.

\addimghere{resources/blogger}{1}{Главная страница платформы blogger}{pic:blogger}

Платформа позволяет быстро авторизоваться через Google-аккаунт и создавать свои блоги.

Присутствует текстовый редактор, показан на рис. \ref{pic:texblogger}, понятный и удобный для пользователя с возможностью поиска совпадений.

\addimghere{resources/texblogger}{1}{Текстовый редактор платформы blogger}{pic:texblogger}

У платформы отсутствует возможность вести командные блоги.

Приведем баллы по критериям в табл. \ref{tabular:rezanabalalogi}.

\begin{table}[H]
	\caption{Нормированная оценка альтернатив}
	\label{tabular:rezanabalalogi}
	\begin{center}
\begin{tabular}{|l|l|l|l|l|l|}
\hline
\multirow{2}{*}{Альтернативы} & \multicolumn{5}{c|}{Критерии}                                                                                                                                                                                                                                                                                                                                                                                    \\ \cline{2-6} 
                              & \begin{tabular}[c]{@{}l@{}}Создание и\\ ведение личных(закрытых)\\ блогов\end{tabular} & \begin{tabular}[c]{@{}l@{}}Создание\\ блогов\\ командой\end{tabular} & \begin{tabular}[c]{@{}l@{}}Ведение \\ блогов \\ командой\end{tabular} & \begin{tabular}[c]{@{}l@{}}Наличие \\ удобного \\ редактора \\ для текста\end{tabular} & \begin{tabular}[c]{@{}l@{}}Сохранить пост, \\ без его\\ публикации\end{tabular} \\ \hline
Medium                        & 80                                                                                     & 100                                                                  & 90                                                                    & 90                                                                                     & 65                                                                              \\ \hline
Хабр                          & 70                                                                                     & 75                                                                   & 65                                                                    & 85                                                                                     & 75                                                                              \\ \hline
Blogger                       & 75                                                                                     & 80                                                                   & 70                                                                    & 80                                                                                     & 70                                                                              \\ \hline
\end{tabular}
	\end{center}
\end{table}


Используя формулу \ref{eq:alta} взвешенной суммы баллов.

\begin{equation}\label{eq:alta}
\sum\nolimits_i^n {W_i  \cdot } B_i 
\end{equation}

Где $B_i$ -- оценка альтернативы по каждому критерию.

Дадим общею оценка альтернативам в табл. \ref{tabular:rezanalogi}.

\begin{table}[H]
	\caption{Нормированная оценка альтернатив}
	\label{tabular:rezanalogi}
	\begin{center}
\begin{tabular}{|l|l|l|l|l|l|l|l|}
\hline
\multirow{2}{*}{Альтернативы} & \multicolumn{5}{c|}{Критерии}                                                                                                                                                                                                                                                                                                                                                                                    & \multicolumn{2}{c|}{\multirow{2}{*}{Итог}} \\ \cline{2-6}
                              & \begin{tabular}[c]{@{}l@{}}Создание и\\ ведение личных(закрытых)\\ блогов\end{tabular} & \begin{tabular}[c]{@{}l@{}}Создание\\ блогов\\ командой\end{tabular} & \begin{tabular}[c]{@{}l@{}}Ведение \\ блогов \\ командой\end{tabular} & \begin{tabular}[c]{@{}l@{}}Наличие \\ удобного \\ редактора \\ для текста\end{tabular} & \begin{tabular}[c]{@{}l@{}}Сохранить пост, \\ без его\\ публикации\end{tabular} & \multicolumn{2}{c|}{}                      \\ \hline
Medium                        & 21,1                                                                                   & 22,4                                                                 & 17,76                                                                 & 15,39                                                                                  & 9,41                                                                            & \multicolumn{2}{l|}{86}                    \\ \hline
Хабр                          & 18,4                                                                                   & 16,77                                                                & 12,83                                                                 & 14,54                                                                                  & 10,86                                                                           & \multicolumn{2}{l|}{73,42}                 \\ \hline
Blogger                       & 19,7                                                                                   & 17,89                                                                & 13,82                                                                 & 13,68                                                                                  & 10,13                                                                           & \multicolumn{2}{l|}{75,26}                 \\ \hline
\end{tabular}
	\end{center}
\end{table}


В результате анализа аналогов получили, что ни одно из готовых платформ не подходит для наших целей, следовательно, нужно спроектировать и разработать информационную систему.

\pagebreak

\subsection{Проектирование информационной системы}

В результате внедрения информационной системы, произойдет переход к оптимальным бизнес-процессам. Покажем на рис. \ref{pic:commpanew} изменения в компонентах системы САО РАН. 

\addimghere{resources/commpanew}{1}{Диаграмма компонентов системы САО РАН}{pic:commpanew}

У лаборатории информатики появился новый интерфейс, информационная система, и решение позволяет нам не нарушить структуру объединения подсистем. 

Далее покажем на рис. \ref{pic:communew} как теперь будут взаимодействовать подсистемы

\addimghere{resources/communicationnew}{1}{Диаграмма компонентов системы САО РАН}{pic:communew}

Появился новый Actor назовем его SuperActor, который представляет из себя ученого или обычного пользователя, который захотел вести свой блог, и решение позволяет нам не нарушить старую структуру взаимодействия подсистем.

Получили, что структура системы САО РАН не будет изменена, так как система имеет большой масштаб и любое изменение в структуре влечет за собой значительные затраты, поэтому появиться только новые возможности в системе. Оптимальный сценарий таков, что лаборатория информатики поддерживает работу информационной системы, к ней обращаться SuperActor, за информацией или наоборот может легко создать новую информацию, остальные подсистемы продолжают свою работу без изменений.

Покажем на рис. \ref{pic:precedent} Use Case диаграмму прецедентов информационной системы, которая отобразить множество выполняемых системой функций и взаимодействия с ней пользователя. 
 
\addimghere{resources/precedent}{1}{Use Case диаграмма прецедентов ИС}{pic:precedent}

Из диаграммы видны действия, которые может выполнять пользователь в системе.

actor:
\begin{itemize}
	\item регистрация, чтобы получить расширенный функционал 
	\item просмотреть блог по ссылке, чтобы по полученной ссылке смотреть посты, для этого действие расширяется предоставлением каталогов постов в блоге.
\end{itemize}

\pagebreak

SuperActor потомок от actor:
\begin{itemize}   
	\item авторизоваться,
	\item создать блог, 
	\item управлять блогами, что включает в себя такие возможности как пригласить автора, создать, сохранить, изменить, опубликовать пост. 
\end{itemize}

\pagebreak
Покажем на рис. \ref{pic:activitinew} диаграмму активности процесса создания новой информации в открытом доступе, вместо создания новых страниц на сайте САО РАН.

\addimghere{resources/activitinew}{1}{Диаграмма активности создания новой информации}{pic:activitinew}

Пользователь системы может перейти по ссылке на блог и открыть нужный ему пост, или перейти в панель блогов, где ему будет предложено, пройти авторизацию или зарегистрироваться, после чего у него будет ряд возможность таких как создать свой блог, выбрать из существующих или уведомление о согласии вступить в команду, далее у пользователя появляется меню с набором действий, таких как создать пост, или изменить существующий, пригласить соавтора, получить краткую информацию о блоге. Таким обзором новая информация окажется в открытом доступе, без увеличения интенсивности информации, между подсистемами в системе.

Изобразим логическую модель базы данных на рис. \ref{pic:bdnew}
\addimghere{resources/bdnew}{1}{Логическая модель отношений сущностей БД}{pic:bdnew}

Логическая модель показывает как взаимодействуют сущности между собой.

Покажем на рис. \ref{pic:erd} ERD диаграмму отношений сущностей, которая отобразить структуру базы данных и их отношение на физическом уровне.

\addimghere{resources/erd}{1}{ERD диаграмма отношений сущностей БД}{pic:erd}

\pagebreak
В таблице Blogs храниться информация о блогах в системе, опишем сущность в табл. \ref{tabular:blogs}

\begin{table}[H]
	\caption{сущность Blogs}
	\label{tabular:blogs}
	\begin{center}
\begin{tabular}{|l|l|l|}
	\hline
	\multicolumn{1}{|c|}{Название поля} & \multicolumn{1}{c|}{\begin{tabular}[c]{@{}c@{}}Тип\\ данных\end{tabular}} & \multicolumn{1}{c|}{Описание}     \\ \hline
	Id                                  & Integer                                                                   & Идентификатор блога               \\ \hline
	Title                               & String                                                                    & Отображаемый заголовок блога      \\ \hline
	Description                         & String                                                                    & Описание блога                    \\ \hline
	Type                                & Integer                                                                   & Тип блога                         \\ \hline
	IsPublic                            & Boolean                                                                   & Флаг публичности блога            \\ \hline
	OwnerId                             & Integer                                                                   & Идентификатор владельца блога     \\ \hline
	CreateDate                          & DateTime                                                                  & Дата создания блога               \\ \hline
	UpdateDate                          & DateTime                                                                  & Дата последнего изменения о блоге \\ \hline
	Name                                & String                                                                    & Уникальное имя блога              \\ \hline
\end{tabular}
	\end{center}
\end{table}



В таблице AuthorInvetes храниться информация для реализации возможности приглашать со авторов в командный блог, опишем сущность в табл. \ref{tabular:AuthorInvetes}

\begin{table}[H]
	\caption{сущность AuthorInvetes }
	\label{tabular:AuthorInvetes}
	\begin{center}
		\begin{tabular}{|l|l|l|}
			\hline
			\multicolumn{1}{|c|}{Название поля} & \multicolumn{1}{c|}{\begin{tabular}[c]{@{}c@{}}Тип\\ данных\end{tabular}} & \multicolumn{1}{c|}{Описание}          \\ \hline
			Id                                  & Integer                                                                   & Идентификатор приглашения              \\ \hline
			SenderId                            & Integer                                                                   & Идентификатор отправителя              \\ \hline
			ReceiverId                          & Integer                                                                   & Идентификатор получателя               \\ \hline
			BlogId                              & Integer                                                                   & Идентификатор блога для приглашения    \\ \hline
			IsApprove                           & Boolean                                                                   & Флаг согласия получателя               \\ \hline
			IsSenderNotified                    & Boolean                                                                   & Флаг просмотра результата отправителем \\ \hline
			CreateDate                          & DateTime                                                                  & Дата создания                          \\ \hline
			ReceiverRespondDate                 & DateTime                                                                  & Дата ответа получателя                 \\ \hline
			SenderViewedDate                    & DateTime                                                                  & Дата просмотра ответа отправителем     \\ \hline
		\end{tabular}
	\end{center}
\end{table}




В таблице BlogAuthor храниться информация для создания командных блогов в системе, создает связь <<многие ко многим>> для сущностей Blogs и Users, опишем сущность в табл. \ref{tabular:BlogAuthor}

\begin{table}[H]
	\caption{сущность BlogAuthor}
	\label{tabular:BlogAuthor}
	\begin{center}
		\begin{tabular}{|l|l|l|}
			\hline
			\multicolumn{1}{|c|}{Название поля} & \multicolumn{1}{c|}{\begin{tabular}[c]{@{}c@{}}Тип\\ данных\end{tabular}} & \multicolumn{1}{c|}{Описание} \\ \hline
			BlogId                              & Integer                                                                   & Идентификатор блога           \\ \hline
			AuthorId                            & Integer                                                                   & Идентификатор пользователя    \\ \hline
		\end{tabular}
	\end{center}
\end{table}




В таблице Users храниться информация о пользователях в системе, опишем сущность в табл. \ref{tabular:Users}

\begin{table}[H]
	\caption{сущность Users}
	\label{tabular:Users}
	\begin{center}
		\begin{tabular}{|l|l|l|}
			\hline
			\multicolumn{1}{|c|}{Название поля} & \multicolumn{1}{c|}{\begin{tabular}[c]{@{}c@{}}Тип\\ данных\end{tabular}} & \multicolumn{1}{c|}{Описание}             \\ \hline
			Id                                  & Integer                                                                   & Идентификатор пользователя                \\ \hline
			FirstName                           & String                                                                    & Имя пользователя                          \\ \hline
			LastName                            & String                                                                    & Фамилия пользователя                      \\ \hline
			Username                            & String                                                                    & Логин пользователя в системе              \\ \hline
			Password                            & String                                                                    & Хэш пароля пользователя в системе         \\ \hline
			Email                               & String                                                                    & E-Mail пользователя                       \\ \hline
			IsActive                            & Boolean                                                                   & Флаг доступности учётной записи           \\ \hline
			IsBlocked                           & Boolean                                                                   & Флаг блокировки учётной записи            \\ \hline
			CreateDate                          & DateTime                                                                  & Дата создания учётной записи              \\ \hline
			UpdateDate                          & DateTime                                                                  & Дата последнего обновления учётной записи \\ \hline
			LastActive                          & DateTime                                                                  & Дата последней активности пользователя    \\ \hline
		\end{tabular}
	\end{center}
\end{table}



\pagebreak
В таблице Posts храниться информация о постах в системе, опишем сущность в табл. \ref{tabular:Posts}

\begin{table}[H]
	\caption{сущность Posts}
	\label{tabular:Posts}
	\begin{center}
		\begin{tabular}{|l|l|l|}
			\hline
			\multicolumn{1}{|c|}{Название поля} & \multicolumn{1}{c|}{\begin{tabular}[c]{@{}c@{}}Тип\\ данных\end{tabular}} & \multicolumn{1}{c|}{Описание}                  \\ \hline
			Id                                  & Integer                                                                   & Идентификатор поста                            \\ \hline
			Title                               & String                                                                    & Заголовок поста                                \\ \hline
			Description                         & String                                                                    & Описание поста                                 \\ \hline
			PostContent                         & String                                                                    & Содержимое поста                               \\ \hline
			IsPublished                         & Boolean                                                                   & Флаг публикации поста                          \\ \hline
			AuthorId                            & Integer                                                                   & Идентификатор автора поста                     \\ \hline
			BlogId                              & Integer                                                                   & Идентификатор блога, к которому относится пост \\ \hline
			CreateDate                          & DateTime                                                                  & Дата создания поста                            \\ \hline
			UpdateDate                          & DateTime                                                                  & Дата последнего изменения поста                \\ \hline
			PublishDate                         & DateTime                                                                  & Дата публикации поста                          \\ \hline
		\end{tabular}
	\end{center}
\end{table}

\pagebreak


Получается входными данными являются:
\begin{itemize}
	\item данные пользователя,
	\item информация для поста.
\end{itemize}

Выходными данными являются:
\begin{itemize}
	\item информация от пользователей.
\end{itemize}  

\subsection{Вывод}
В ходе выполнения проектирования были сформированы критерии для информационной системы, были оценены по этим критериям готовые аналоги решения. Построены диаграммы UML, такие, как Use Case диаграмма прецедентов, для отображения выполняемых функций системы, диаграмма активности показывающая как происходит процесс создания информации для пользователя, и взаимодействие пользователя с разрабатываемой системой. Разработана структура хранения данных, показана их логическая и физическая модель, с описанием. Указали входные и выходные данные, для информационной системы.

\pagebreak