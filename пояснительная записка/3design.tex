\section{Проектирование информационной системы}

\subsection{Формирование параметров и требований к ИС}
%Представим перечни, которыми должны обладать разрабатываемая система.
Система должна выполнять перечень задач:
\begin{itemize}
	\item создание аккаунта любому пользователю, который так же может быть вне системы САО РАН, 
	\item создание блогов,
	\item создание и редактирование постов(оформление статей в блогах),
	\item приглашать других пользователей в качестве авторов,
	\item видеть список постов внутри блога,
	\item иметь возможность сохранить пост, чтобы позже его опубликовать. 
\end{itemize}

Из перечня задач сформируем требования к разрабатываемой системе:
\begin{itemize}
	\item создание и видение личных блогов,
	\item создание блогов командой,
	\item введение блогов командой,
	\item наличие удобного редактора для текста,
	\item сохранить пост, без его публикации. 
\end{itemize}

\subsection{Сравнение аналогов}
В сети Интернет уже давно развиваются разные системы с блогами, наиболее популярными системами являются Medium, Хабр и Blogger, рассмотрим их более подробно.

\pagebreak
\subsubsection{Medium}
Medium платформа для социальной журналистики, позволяет создавать аккаунты, писать стати так называемые <<Story>>, коллективные блоги называемые <<Publications>>, покажем на рисунке \ref{pic:medium} как выглядит главная страница платформы.

\addimghere{resources/medium}{0.9}{Главная страница Medium}{pic:medium}

Присутствует текстовый редактор, показан на рисунке \ref{pic:textmedium}, среднего качества. Нужно оформлять платную подписку, чтобы получить более полный функционал платформы.

\addimghere{resources/textmedium}{0.9}{Текстовый редактор платформы Medium}{pic:textmedium}

\subsubsection{Хабр}

Хабр платформа с тематическими командными блогами, где каждый может написать свою статью, покажем на рисунке \ref{pic:habr} как выглядит главная страница платформы.

\addimghere{resources/habr}{0.9}{Главная страница Хабр}{pic:habr}

Платформа позволяет зарегистрироваться и писать статьи по своей тематике, так называемые <<Хабы>>. От новых пользователей стати проходят модерацию. После того как пользователь покажет себя с хорошей стороны, ему станет доступен полный доступ, и новые статьи от него будут публиковаться сразу.

\pagebreak

Присутствует текстовый редактор, показан на рисунке \ref{pic:texthabr}, с возможностью форматирования специальный разметкой <<Markdown>>, что делает его понятным и удобным для пользователя.

\addimghere{resources/texthabr}{0.9}{Текстовый редактор платформы Хабр}{pic:texthabr}

Пользователь не может вести отдельный личный блог, без попадания его статьи в общие блоги. Ограничен список тематик на которые можно писать статьи из-за модерации, так как площадка позиционирует себя, по большей части, как техно-научная.

\pagebreak

\subsubsection{Blogger}

Blogger платформа специально предназначенная для ведения блогов, покажем на рисунке \ref{pic:blogger} как выглядит главная страница платформы.

\addimghere{resources/blogger}{1}{Текстовый редактор платформы blogger}{pic:blogger}

Платформа позволяет быстро авторизоваться через Google аккаунт 

\pagebreak