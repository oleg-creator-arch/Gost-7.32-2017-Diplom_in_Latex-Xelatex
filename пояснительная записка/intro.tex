\anonsection{ВВЕДЕНИЕ}
Актуальность исследования обусловлена необходимостью улучшения обмена информацией с другими подсистемами в общей системе САО РАН. Для решения проблемы восприятия информации была разработана информационная система <<>> . Проблема возникла, потому что в современном мире фундаментальная и прикладная наука, на ряду с увеличением деятельности пользователей в интернете, приносит большие объемы данных, из-за чего стало накапливаться значительное количество информации, а огромное количество данных создает проблемы для их анализа и интерпретации. Последствия «информационного взрыва» приводят в конечном счете к развитию направления Big Data. Направление Big Data выступает альтернативой увеличения вычислительных мощностей, тем что дает новые информационные концепции и технологии, одной из которых является когнитивная визуализация данных.

Целью данной работы является, исследования особенности создания web-сервисов для работы с системами интеллектуального анализа больших данных в сетевых и облачных средах, исследовать возможности удалённого сетевого применения методики когнитивного анализа Big Data, предложить способы доступа к разнородным массивам данных, относящихся к одной предметной области, но которые получены из различных источников различными средствами, различными методами и хранятся в разных форматах под разными СУБД. В качестве программных систем когнитивной визуализации освоить и использовать программные системы SpaceWalker(SW) реализована\right)  на базе Java программная система когнитивной визуализации, создающая динамический 3D образ, которая позволяет оперативно проводить многомерный когнитивный анализ многомерных данных произвольной природы, и SpaceHedgehog(SH) методика и технология когнитивной 6D визуализации многомерных данных, разработанная на основе исследования и развития технологии когнитивной визуализации, которая предоставляет возможность одновременно визуально воспринимать и анализировать в динамике шесть произвольных параметров многомерного объекта, проанализировать и предложить методику и программные средства разработки для них web-сервисов на различных платформах или мультиплатформенной основе, разработать web-сервис для использования программных комплексов когнитивного анализа Big Data, обеспечивающий не только быстрый и надёжный доступ к системам, но и релевантную визуальную интерактивность процесса когнитивного анализа данных в приемлемой для систем когнитивной визуализации форме, исследовать применимость и эффективность разработанного сервиса для систем когнитивного анализа данных.

В соответствии с поставленной целью были определены следующие задачи:
\begin{itemize}
    \item изучить текущие состояние объекта,
    \item сформировать математическую модель объекта,
    \item проанализировать математическою модель и выявить причину проблемы и предложить улучшение объекта, 
    \item спроектировать web-сервис по улучшенной модели объекта,
    \item реализовать web-сервис,
    \item оценить социальную значимость разработки,  
    \item оценить технико-экономическое обоснование разработки,
    \item оценить безопасность и экологичность разработки,
\end{itemize}

 Практическая ценность работы заключается в создании web-сервиса для когнитивного анализа распределённых гетерогенных данных

\pagebreak