\anonsection{ВВЕДЕНИЕ}
Актуальность исследования обусловлена необходимостью улучшения обмена информацией с другими подсистемами в общей системе САО РАН. Для решения проблемы восприятия информации была разработана информационная система. Проблема возникла, потому что в современном мире науки, на ряду с увеличением деятельности с пользователями в интернете, приходится работать с большем объемом данных, из-за чего стало накапливаться значительное количество информации, что создает проблемы, для их восприятия, анализа и интерпретации, в составлении отчетов о деятельности. 

Информационная система будет предназначена для увлечения удобства работы пользователю с целью получения ему нужной информации, и составлению отчетов для улучшения взаимодействия с другими подсистемами.

Целью работы является увеличить количество воспринимаемой информации пользователями. 

Существуют информационные системы, которые помогают автоматизировать процессы анализа и составления отчетов, так же большинство из них обладает ещё и дополнительным функционалом по мимо этого. Основном не достатком этих информационных систем является, что не всегда как раз таки нужен весь этот обширный функционал необходим для конкретного решения, и второй недостаток, чего-то может не хватить, из-за чего понадобиться тратить ресурсы на доработку, или даже окажется, что не будет такой возможности. После анализа САО РАН было решено создать информационную систему для лаборатории информатики в виде web-приложения.

Практическая ценность работы заключается в создании информационной системы, для лаборатории информатики, разработка которой будет решать определенный ряд задач:
\begin{itemize}
	\item увеличение количества информации воспринятой пользователем,
	\item создание базы данных для удобного доступа к данным,
	\item снижение ресурсов на оформление отчетов,
	\item минимизации человеческого фактора.
\end{itemize}


В соответствии с поставленной целью были определены следующие задачи, для её достижения:
\begin{itemize}
    \item изучить текущие состояние подсистемы в системе,
    \item сформировать математическую модель объекта,
    \item проанализировать математическою модель и выявить причину проблемы и предложить улучшение объекта, 
    \item спроектировать информационную систему по улучшенной модели объекта,
    \item реализовать информационную систему,
    \item оценить социальную значимость разработки,  
    \item оценить технико-экономическое обоснование разработки,
    \item оценить безопасность и экологичность разработки.
\end{itemize}

 
\pagebreak