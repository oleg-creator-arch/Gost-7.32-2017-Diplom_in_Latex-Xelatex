\section{Оптимизация и реинжиниринг бизнес-процессов}
\subsection{Оптимизация математической модели}
Входе анализа модели САО РАН выявили бизнес-процесс, который влияет на параметр $q_A$ интенсивность информации, покажем его относительно параметра $q_A$ на рисунке \ref{pic:graph}.

\addimghere{resources/activiti}{1}{диаграмма активности создания новой страницы}{pic:graph}

За счёт того что вспомогательно-техническому подразделению приходиться обращаться за информацией в разные подсистемы, а после передавать информацию в лабораторию информатики, что бы там создали новую страницу на сайте, для того чтобы пользователи веб-сайта могли её просмотреть, получается, что $q_A \rightarrow \infty$ увеличивается параметр интенсивности информации. Для того чтобы оптимизировать процесс, предлагается вариант решения, который уменьшив интенсивность информации без уменьшения её количества, для этого следует использовать информационную систему, к которой был бы доступ любой подсистеме в САО РАН, а так же самим пользователям, которые захотят создать и вести свой блог.
%Написать мат модель выгоды ИС взять из курсовой
Необходимо вычислить оптимальное решение из двух вариантов, первый вариант оставить уже рабочею модель, а второй вариант отказаться от перемещения одинаковой информации по системе, потому что входе этих перемещений теряется часть знаний о ней и увеличивается интенсивность обмена информации между подсистемами в системе за счет информационной системы.

Исследуемая операция – создание страницы с информацией на сайте, система – варианты размещения информации на сайте, показатель исхода операции – число интенсивности информации $q_A$ (дискретная величина). Стоимость системы без изменений 30 у.е. стоимость новой системы 40 у.е. Допустим вероятность хорошего спроса на информацию -- 70\%, слабого спроса на информацию -- 30\%. При хорошем спросе на первый вариант количество информации которую усвоит пользователь будет составлять 100 у.е. при плохом 20 у.е. Для второго варианта при хорошем спросе усвоенной информации у пользователя будет 150 у.е. при плохом спросе 60 у.е.

Полезность рассчитаем, как разность между усвоенной информацией и затратами на систему. Следовательно, значения полезность для первого варианта, при хорошем спросе на систему 70 у.е. при плохом спросе -10 у.е. для второго варианта при хорошем спросе 110 у.е. при плохом 20 у.е.
Из приведённых данных дадим оценку эффективности для вариантов решений, за счет того что найдем сумму произведений вероятности исходы на полезность. Для первого варианта эффективность решения будет равна -3 у.е. для второго варианта решения 6 у.е.

Исходя из оценки эффективности, можно сделать вывод, что при втором варианте решения будет наименьшее число интенсивности информации, то есть значения параметра $q_A$ будет меньше. Поэтому следует воспользоваться предложенным решением и перейти к более оптимизированной системе(вариант размещения информации на сайте). 

 
\pagebreak