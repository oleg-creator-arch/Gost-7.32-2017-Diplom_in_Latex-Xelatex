\section{Оптимизация и реинжиниринг бизнес-процессов}
\subsection{Оптимизация математической модели}
В ходе анализа предметной области выявили проблемный бизнес-процесс, получается, что $q_A \rightarrow \infty$ увеличивается параметр интенсивности информации. Для того чтобы оптимизировать процесс, предлагается вариант решения, который уменьшит интенсивность информации, с сохранением её количества, для этого нужно использовать информационную систему. К информационной системе должен быть доступ любой подсистеме в САО РАН, а так же самим пользователям, которые захотят создать и вести свой блог.
%Написать мат модель выгоды ИС взять из курсовой
Одной из задач САО РАН является обеспечить максимальную эффективность усвоения информации пользователем, при этом сама ценность такой информации для пользователя может меняться. Следовательно, рассмотрим пользователя, у которого цель получать информацию, как пассивную систему $B$, а пользователя, который оставляет информацию как активную систему $A$. Получаем задачу, что в математической модели предприятия должны максимизировать ценность информации, покажем это на формуле \ref{eq:matmodelqamax}: 
\begin{equation}\label{eq:matmodelqamax}
\bar \sigma  = \int\limits_0^\tau  {\left[ {1 - p\left( {q_A \left( {\upsilon _A ,\upsilon _B \left( I \right)} \right)} \right)} \right]} q_A \left( {\upsilon _A ,\upsilon _B \left( I \right)} \right)dt \to \mathop {\max }\limits_{\upsilon _A } 
\end{equation}

Где $\upsilon _A$ -- ценность информации. 

Перепишем задачу через управляющий параметр интенсивности информации $q_A$ покажем это на формуле \ref{eq:matmodelqamaxmum}: 
\begin{equation}\label{eq:matmodelqamaxmum}
\int\limits_{I_0 }^{I_F } {\frac{{1 - p(q_A )}}{{p(q_A )}}} dI \to \mathop {\max }\limits_{q_A } 
\end{equation} 

Где $q_A$ -- интенсивность информации.

Из чего следует, что площадь на графике(количество усвоенной информации пользователем) должна изменяться при изменении интенсивности $q_A$. Покажем на графике, как измениться значение усвоенной информации пользователем, за счет изменения интенсивности в два раза после оптимизации (как должно быть),  $q_A$ в меньшую сторону (рис. \ref{pic:UPmathgraphic}). 
\addimghere{resources/UPmathgraphic}{0.5}{Сколько усвоить информации пользователь после оптимизации}{pic:UPmathgraphic}
Заметно, что площадь(выделено черным) увеличилась с уменьшением интенсивности. Покажем полученное значение, на рис. \ref{pic:reshq} показан неопределённый интеграл.

\addimghere{resources/reshq}{1}{Не определный интеграл после оптимизации}{pic:reshq} 

Подставим наши ограничения, пользователю предоставляться та же информация, следовательно, ограничения такие же, что на сайте общее количество доведенной до пользователя информации фиксировано $I\left( \tau  \right) = I_F = 0 МБ$, и у пользователя уже иметься некоторая начальная информация $I\left( 0 \right) = I_0 = 0 МБ$. 

Численное значение воспринимаемое пользователем $I = 6.39$ МБ (рис. \ref{pic:answarqq}).

\addimghere{resources/answarqq}{1}{Численое значение сколько воспринимает информации пользователь после оптимизации}{pic:answarqq}

Получили, что $ \frac{{I'}}{I} = \frac{{6,39}}{{1,72}} = 3,72 $ раз увеличилось значение информации воспринятое пользователем.






\subsection{Сравнительный анализ решений}
Проведем математический анализ аналогов решений, методом количественного оценивания систем <<оценка сложных систем в условиях риска на основе функции полезности>>, согласно учебному пособию В. С. Анфилатова <<операции, которые выполняться в рисковых условиях являются вероятностными>> [2].

Необходимо вычислить оптимальное решение из двух вариантов, первый вариант оставить уже рабочею модель, а второй вариант отказаться от перемещения одинаковой информации по системе, потому что входе этих перемещений теряется часть знаний о ней и увеличивается интенсивность обмена информации между подсистемами в системе за счет информационной системы.

Исследуемая операция – создание страницы с информацией на сайте, система – варианты размещения информации на сайте, показатель исхода операции – число интенсивности информации $q_A$ (дискретная величина). Стоимость системы без изменений 30 у.е. стоимость новой системы 40 у.е. Допустим вероятность хорошего спроса на информацию -- 70\%, слабого спроса на информацию -- 30\%. При хорошем спросе на первый вариант количество информации которую усвоит пользователь, будет составлять 100 у.е. при плохом 20 у.е. Для второго варианта при хорошем спросе усвоенной информации у пользователя будет 150 у.е. при плохом спросе 60 у.е.

Полезность рассчитаем, как разность между усвоенной информацией и затратами на систему. Следовательно, значения полезность для первого варианта, при хорошем спросе на систему 70 у.е. при плохом спросе -10 у.е. для второго варианта при хорошем спросе 110 у.е. при плохом 20 у.е.
Из приведённых данных дадим оценку эффективности для вариантов решений, за счет того что найдем сумму произведений вероятности исходы на полезность. Для первого варианта эффективность решения будет равна -3 у.е. для второго варианта решения 6 у.е.

Исходя из оценки эффективности, можно сделать вывод, что при втором варианте решения будет наименьшее число интенсивности информации, то есть значения параметра $q_A$ будет меньше. Поэтому следует воспользоваться предложенным решением и перейти к более оптимизированной системе(вариант размещения информации на сайте). 

Из-за возможных рисков следует воспользоваться вторым вариантом решения, с сохранением старой работающей системой.

\pagebreak

\subsection{Реинжиниринг бизнес-процессов}

В результате внедрения информационной системы, произойдет переход к оптимальным бизнес-процессам. Покажем на рис. \ref{pic:bpmn2} изменения в компонентах системы САО РАН.
\addimghere{resources/bpmn2}{0.65}{Диаграмма BPMN системы САО РАН после оптимизации}{pic:bpmn2}

Перечень изменений бизнес-процессов таков, что теперь не нужно подчиняться закону стандартов РФ и уставу САО РАН, что бы опубликовать свои знания по исследованию, в открытый доступ для клиента, так же теперь за счёт пару кликов информация сможет дойти до клиента, за счёт создания своевременных блогов, что привлечёт большее внимание пользователей.
У лаборатории информатики появился новый интерфейс, информационная система, и решение позволяет нам не нарушить структуру объединения подсистем, решение позволяет нам не нарушить старую структуру взаимодействия подсистем, и у них остался доступ к вспомогательному подразделению.


\subsection{Вывод}
Было предложено решение проблемы, оптимизировать предметную область, за счёт другой системы размещения информации в общем доступе. Проведен математический анализ решений, и выбор самого эффективного из них с учетом риска. Моделирование на диаграммах UML оптимизированных процессов, с описанием их изменений. 
\pagebreak