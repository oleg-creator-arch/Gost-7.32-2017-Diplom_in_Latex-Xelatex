\section{Социальная значимость разработки}
Существуют информационные системы, которые помогают автоматизировать процессы анализа и составления блогов, так же большинство из них обладает ещё и дополнительным функционалом помимо этого. Основном не достатком этих информационных систем является, что не всегда как раз таки необходим весь этот обширный функционал, для конкретного решения, и второй недостаток, чего-то может не хватить, из-за чего понадобиться тратить ресурсы на доработку, или даже окажется, что не будет такой возможности.
Если сравнить с аналогами информационная система не нагружена, и сохраняет оптимальность, следовательно, задача, по проектированию и разработки информационной системы, является обоснованной и необходимой с социальной точки зрения.

Ценность работы заключается в создании информационной системы, для лаборатории информатики, разработка которой будет решать определенный ряд задач:
\begin{itemize}
	\item увеличение количества информации, которая будет воспринята пользователем,
	%\item создание информационной системы  для удобного доступа к данным,
	\item снижение ресурсов на оформление статей,
	\item минимизации человеческого фактора.
\end{itemize}

Через информационную систему регулярно за информацией обращаются разного рода пользователи, которые внутри её системы и вне. Социальная значимость разработки заключается в том, что появилась возможность оформление блогов о своей деятельности, и возможностью делиться своим опытом, такая возможность привлечет людей к науке, что вызовет увеличение эрудированности в обществе. 

.  

\pagebreak