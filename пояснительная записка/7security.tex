\section{Безопасность и экологичность разработки}
Что бы эффективность труда была высокой, нужно учитывать условия работы пользователя информационной системы. В основном его работа это интеллектуальная деятельность, поэтому для оценки условий труда нужно использовать показатели напряженности трудового процесса. 

Напряженность труда пользователя оценивается методом анализа трудовой деятельности и её структуры, посредством наблюдения всего рабочего дня в течение определённого времени, к примеру на протяжении одной недели. За это время анализируются все производственные факторы, как благоприятные, так и неблагоприятные, которые происходят в процессе деятельности. Полученные показатели трудового процесса имеют качественную и количественную выраженность, а также делятся на виды нагрузок, таких как интеллектуальные, эмоциональные режимные, монотонные, сенсорные. 

По степени опасности и вредоносности условия труда делятся на четыре класса:
\begin{itemize}
    \item оптимальные (1 класс) –- сохраняется здоровье работника, поддержание высокой степени работоспособности,
    \item допустимые (2 класс) -- вредные факторы не превышают допустимые значения. В течение регламентированного отдыха организм полностью восстанавливается,
    \item вредные (3 класс) -- вредные факторы превышают допустимые значения, и вызывают негативные воздействия на организм,
    \item опасные (4 класс) -- воздействие вредных факторов создает угрозу для жизни, и высок риск развития профессиональных поражений организма.
\end{itemize}

Проведем оценку напряженности трудового процесса: интеллектуальные нагрузки в таблице \ref{tabular:opasInt}, сенсорные нагрузки в таблице \ref{tabular:opasSen}, эмоциональные нагрузки в таблице \ref{tabular:opasAmoz}, монотонность нагрузок в таблице \ref{tabular:opasMon}, режим работы в таблице \ref{tabular:opasWork}

\begin{table}[H]
\caption{Оценка интеллектуальной нагрузки}
\label{tabular:opasInt}
\begin{center}
\begin{tabular}{|l|l|c|} 
\hline
Фактор                                                                                        & Заключение                                                                                                                                                    & \multicolumn{1}{l|}{Оценка}  \\ 
\hline
содержание работы                                                                             & решении задач,с единоличным~принятием решения                                                                                                                 & 2                            \\ 
\hline
\begin{tabular}[c]{@{}l@{}}восприятие информации\\их оценка\end{tabular}                      & \begin{tabular}[c]{@{}l@{}}восприятие информации с последующей коррекцией\\действий и выполняемых операций~\end{tabular}                                      & 2                            \\ 
\hline
\begin{tabular}[c]{@{}l@{}}распределение функций \\по степени сложности задания~\end{tabular} & \begin{tabular}[c]{@{}l@{}}выполнение конкретного задания осуществляется с обработкой,\\и выполнением с последующей проверкой выполнения задания\end{tabular} & 2                            \\ 
\hline
характер выполняемой работы                                                                   & \begin{tabular}[c]{@{}l@{}}работа протекает по строго установленному графику \\с возможной его коррекцией по мере необходимости\end{tabular}                  & 2                            \\
\hline
\end{tabular}
\end{center}
\end{table}

\begin{table}[H]
\caption{Оценка сенсорной нагрузки}
\label{tabular:opasSen}
\begin{center}
\begin{tabular}{|l|l|c|} 
\hline
Фактор                                                                                                                                        & Заключение                                & \multicolumn{1}{l|}{Оценка}  \\ 
\hline
длительность
сосредоточенности
наблюдения                                                                                                     & 26-50\%                                   & 2                            \\ 
\hline
\begin{tabular}[c]{@{}l@{}}плотность сигналов и сообщений\\за 1 часработы\end{tabular}                                                         & меньше 100                                & 1                            \\ 
\hline
\begin{tabular}[c]{@{}l@{}}число производственных объектов\\одновременного наблюдения\end{tabular}                                             & до 5 объектов~                            & 1                            \\ 
\hline
\begin{tabular}[c]{@{}l@{}}размер объекта различения\\при длительности сосредоточенного \\внимания (\% от времени смены)\end{tabular}         & 5-1,1 мм 50\%                             & 2                            \\ 
\hline
\begin{tabular}[c]{@{}l@{}}Работа с оптическими приборами\\при длительности\\сосредоточенного \\наблюдения (\% от времени смены)\end{tabular} & Отсутствует                               & 1                            \\ 
\hline
Наблюдение за экраном видеотерминала (ч в смену)                                                                                              & 4-6 часов                                 & 3,1                          \\ 
\hline
Нагрузка на слуховой анализатор                                                                                                               & помех нет, разборчивость слов равна 100\% & 1                            \\ 
\hline
\begin{tabular}[c]{@{}l@{}}Нагрузка на голосовой аппарат\\(суммарное количество часов\\наговариваемых в неделю)\end{tabular}                  & меньше 16 часов                           & 1                            \\
\hline
\end{tabular}
\end{center}
\end{table}

\begin{table}[H]
\caption{Оценка эмоциональной нагрузки}
\label{tabular:opasAmoz}
\begin{center}
\begin{tabular}{|l|l|c|} 
\hline
Фактор                                                                                                                  & Заключение                                                                                                                  & \multicolumn{1}{l|}{Оценка}  \\ 
\hline
\begin{tabular}[c]{@{}l@{}}степень ответственности\\за результат собственной деятельности.\\Значимость ошибки\end{tabular} & \begin{tabular}[c]{@{}l@{}}случае допущенной ошибки\\дополнительные усилия\\только со стороны самого работника\end{tabular} & 1                            \\ 
\hline
степень риска для
собственной жизни                                                                                     & отсутствует                                                                                                                 & 1                            \\ 
\hline
\begin{tabular}[c]{@{}l@{}}степень ответственности \\за безопасность других лиц\end{tabular}                            & отсутствует                                                                                                                 & 1                            \\
\hline
\end{tabular}
\end{center}
\end{table}

\begin{table}[H]
\caption{Оценка монотонности нагрузки}
\label{tabular:opasMon}
\begin{center}
\begin{tabular}{|l|l|c|} 
\hline
Фактор                                                                                                                                                       & Заключение  & \multicolumn{1}{l|}{Оценка}  \\ 
\hline
\begin{tabular}[c]{@{}l@{}}число элементов, \\необходимых для реализации \\простого задания или\\многократно повторяющихся операций\end{tabular}             & 5-10        & 2                            \\ 
\hline
\begin{tabular}[c]{@{}l@{}}продолжительность (с) выполнения простых \\производственных заданий или\\повторяющихся операций\end{tabular}                      & 5-100с      & 2                            \\ 
\hline
\begin{tabular}[c]{@{}l@{}}время активных действий\\(в \% к продолжительности смены)\end{tabular}                                                            & больше 25\% & 1                            \\ 
\hline
\begin{tabular}[c]{@{}l@{}}монотонность производственной обстановки\\(время пассивного наблюдения за ходом~\\техпроцесса в \% от времени смены)\end{tabular} & меньше 70\% & 1                            \\
\hline
\end{tabular}
\end{center}
\end{table}

\begin{table}[H]
\caption{Оценка нагрузки режима работы}
\label{tabular:opasWork}
\begin{center}
\begin{tabular}{|l|l|l|} 
\hline
\multicolumn{1}{c|}{Фактор}                                                                                                                          & \multicolumn{1}{c|}{Заключение}                                                                                    & \multicolumn{1}{c|}{Оценка}  \\ 
\hline
фактическая продолжительность рабочего дня                                                                                      & 7-8 ч                                                                                         & 1                            \\ 
\hline
сменность работы                                                                                                                & односменная работа
(дневная)                                                                  & 1                            \\ 
\hline
\begin{tabular}[c]{@{}l@{}}наличие регламентированных перерывов и \\их продолжительность (без обеденного перерыва)\end{tabular} & \begin{tabular}[c]{@{}l@{}}перерывы регламентированы,\\больше 7\% рабочего времени\end{tabular} & 1                            \\
\hline
\end{tabular}
\end{center}
\end{table}

Посчитаем количество показателей в каждом классе результат приведен в таблице \ref{tabular:opasRez}

\begin{table}[H]
\caption{Оценка нагрузки режима работы}
\label{tabular:opasRez}
\begin{center}
\begin{tabular}{|c|l|l|l|l|l|} 
\hline
\multirow{2}{*}{Показатели}            & \multicolumn{5}{c|}{Класс условий труда    }  \\ 
\cline{2-6}
                                       & 1  & 2 & 3,1 & 3,2 & 3,3                      \\ 
\hline
количество показателей в каждом классе & 16 & 8 & 1   & 0   & 0                        \\
\hline
\end{tabular}
\end{center}
\end{table}

От 1 до 5 показателей отнесены к 3.1 и/или 3.2 степеням вредности, а остальные показатели имеют оценку 1-го и/или 2-го классов, следовательно устанавливается допустимый (2 класс). Сотруднику необходимо соблюдать профилактические мероприятия, чтобы предотвратить негативные последствия профессиональной деятельности

Сотрудник работает мало подвижно, что влечет за собой риск возникновения синдрома длительных статических нагрузок (СДСН).

Форма СДСН может возникать из-за ухудшения кровообращения, чтобы их предотвратить, или хотя бы уменьшить появление, нужно правильно организовать рабочее место, также это поможет снизить развитие искривления позвоночника.

Опишем организованное рабочее место. Высота рабочей поверхности стола должна регулироваться в пределах 680 - 800 мм; при отсутствии такой возможности высота рабочей поверхности стола должна составлять 725мм. Рабочий стул должен быть регулируемым по высоте и углам наклона 79 сиденья и спинки, а также расстоянию спинки от переднего края сиденья. Регулировку высоты поверхности сиденья в пределах 400-550 мм и углам наклона вперед до 15 градусов, и назад до 5 градусов; высоту опорной поверхности спинки 300 мм, ширину не менее 380 мм; стационарные или съемные подлокотники длиной не менее 250 мм, шириной 50-70 мм. Клавиатуру следует располагать на поверхности стола на расстоянии 100-300 мм от края, обращенного к пользователю или на специальной, регулируемой по высоте рабочей поверхности. Конструкция и размеры рабочего места приведены на рисунке \ref{pic:rabmesto}.
\addimghere{resources/rabmesto.png}{0.7}{Конструкция и размеры рабочего места}{pic:rabmesto}
При работе оператора с компьютером следует прерывать работу один раз в час на пятнадцать минут, занимаясь в это время гимнастическими упражнениями, или переключиться на другой вид деятельности, где смениться положение и уменьшиться зрительная активность.

Продолжительна работа за компьютером быстро утомляться глаза и развивается синдром сухого глаза. Это происходит, потому что испаряемость слезы из-за излучений от монитора повышается, и человек реже моргает, смотря в монитор.
Для предотвращения ухудшения зрения, при работе за монитором следует придерживаться некоторых рекомендаций: 
\begin{itemize}
    \item монитор должен быть установлен на расстоянии 35-65 см от глаз, а
    центр экрана - на 20-25 см ниже уровня глаз,
    \item дисплей не должен быть повернут экраном в сторону окна. В случае
его расположения возле окна, необходимо расположить его перпендикулярно
стеклу,
    \item свет от осветительных ламп не должен падать на дисплей с углом
более 60 градусов от вертикали,
    \item освещенность рабочего места необходимо поддерживать в пределах
порядка 170-250 Лк,
    \item при освещении рабочего места неприемлемо использование
мигающих источников света,
    \item интерьер, на фоне которого установлен дисплей, должен быть
неярким, не бросающимся в глаза. Соотношение яркости экрана и окружения
не должно превышать 3: 1,
    \item необходимы перерывы через каждый час работы - 5-10 минут. Допускается непрерывная работы с компьютером 2 часа,
    \item использование специальных очков для работы за компьютером с прогрессивными линзами.
\end{itemize}


Информационная система на прямую не может принести экологическую угрозу окружающей среде. При использовании разработки происходит сбережение человеческих и временных ресурсов. 
Использующиеся компьютерная техника должна придерживаться стандарта RoHS.  

Что бы обеспечить электробезопасность должны соблюдаться следующие меры, электрическое разделение сетей, обеспечение недоступности токоведущих частей, зануление, защитное заземление, контроль и профилактика поврежденной изоляции.

Оператору требуется быть психологически подготовленным, к необходимости быть сконцентрированным на протяжении длительного наблюдения за работой перед монитором, во избежание ошибок пользователя.  

\pagebreak