\section{Технико-экономическое обоснование разработки}
\subsection{Описание целесообразности проектирования с точки зрения \\объекта} 
Внедрённая информационная система позволит получать информацию удобней и с меньшим количеством ошибок, так как основные функции информационной системы сбор, хранение и предоставление данных, следовательно, сократиться их время получения, расшириться возможность проведения различных аналитических действий с получаемой информацией.

\begin{comment}
\subsection{Расчет затрат, до и после внедрения web-сервиса,\\ на функционирование объекта}
Для определения эффективности сравним результаты расходов на функционирования объекта до автоматизации и после автоматизации.

Расходы на различные виды выплат сотрудникам рассчитываются по формуле (\ref{eq:Zp}):
\begin{equation}\label{eq:Zp}
Z = \sum_{i}^{}{\left(n_{i}\tilde{z_{i}}\left(1+\frac{a_{c}}{100} \right) \right)}
\end{equation}

$Z$ -- суммарный расход на выплаты,

$n_{i}$ -- численность персонала,

$\tilde{z_{i}}$ -- среднегодовая заработная плата работника,

$a_{c}$ -- процент отчислений на социальное страхование, пенсионный фонд и фонд стабилизации.
%%%%%%%%%%%%%%%%
\newpage
%%%%%%%%%%%%%%%%
В объекте задействованы следующие сотрудники:
\begin{itemize}
    \item Ответственный за хранение данных: количество 1; среднегодовая заработная плата работника 360000; процент отчислений 72000,
    \item Ответственный за передачу данных на объект: количество 1; среднегодовая заработная плата работника 400000; процент отчислений 80000.
\end{itemize}
Совершим расчет по приведённым данным: 
\begin{flushleft}
\begin{math}
Z = \sum_{i}^{}{\left(n_{i}\tilde{z_{i}}\left(1+\frac{a_{c}}{100} \right) \right)} = \\ = \left( 1\cdot 360000 \left(1+\frac{72000}{100} \right)\right) + \left( 1\cdot 400000 \left(1+\frac{80000}{100} \right)\right) = \\ = 320400000
\end{math}(рублей)\end{flushleft} 

Повышение производительности происходит, потому что время доступа к нужным данным сократилось, из-за использования новой разработки. Повышение производительности труда $p_{j}$ (в процентах) определяется по формуле (\ref{eq:Proz}):
\begin{equation}\label{eq:Proz}
p_{j} = \frac{\Delta T_{j}}{t_{j}-\Delta T_{j}} \cdot 100\%
\end{equation}

$p_{j}$ -- суммарный расход на выплаты,

$\Delta T_{j}$ --  ;

$\tilde{z_{i}}$ -- среднегодовая заработная плата работника;

$a_{c}$ -- процент отчислений на социальное страхование, пенсионный фонд и фонд стабилизации.
\end{comment}
\subsection{Расчет затрат на разработку}
Что бы рассчитать затраты на необходимые работы, для реализации разработки, определим продолжительность работ.
Продолжительность работ можно определить, или по нормативам, или рассчитать среднее время продолжительности работ по формуле (\ref{eq:Dlrab}):

\begin{equation}\label{eq:Dlrab}
T_{0} = \frac{3T_{min} + 2T_{max}}{5}
\end{equation}

$T_{0}$ -- ожидаемая длительность работ,

$T_{min}$ -- наименьшая длительность работ,

$T_{max}$ -- наибольшая длительность работ.

Совершим расчет по приведённым данным, расчеты ожидаемой длительности работ до разработки  приведены в таблице \ref{tabular:DoTime}, и расчеты ожидаемой длительности работ после разработки приведены в таблице \ref{tabular:PosleTime}.

\begin{table}[H]
\caption{Ожидаемая длительность работ до внедрения разработки }
\label{tabular:DoTime}
\begin{center}
\begin{tabular}{|l|c|c|c|}
\hline
\multicolumn{1}{|c|}{\multirow{2}{*}{Наименование работы}} & \multicolumn{3}{c|}{Длительность работы (час)}                                       \\ \cline{2-4} 
\multicolumn{1}{|c|}{}                                     & \multicolumn{1}{l|}{min} & \multicolumn{1}{l|}{max} & \multicolumn{1}{l|}{ожидаемая} \\ \hline
поиск необходимой информации                               & 4                        & 6                        & 4,8                              \\ \hline
получение информации                                       & 3                        & 5                        & 3,8                              \\ \hline
сохранение информации                                      & 2                        & 5                        & 3,2                              \\ \hline
\multicolumn{3}{|l|}{Итог:}                                                                                      & 11,8                               \\ \hline
\end{tabular}
\end{center}
\end{table}

\begin{table}[H]
\caption{Ожидаемая длительность работ после внедрения разработки }
\label{tabular:PosleTime}
\begin{center}
\begin{tabular}{|l|l|l|l|}
\hline
\multicolumn{1}{|c|}{\multirow{2}{*}{Наименование работы}} & \multicolumn{3}{c|}{Длительность работы (час)}                                       \\ \cline{2-4} 
\multicolumn{1}{|c|}{}                                     & \multicolumn{1}{l|}{min} & \multicolumn{1}{l|}{max} & \multicolumn{1}{l|}{ожидаемая} \\ \hline
поиск необходимой информации                               & 3                        & 5                        & 3,8                              \\ \hline
получение информации                                       & 0,5                        & 1                        & 0,7                              \\ \hline
сохранение информации                                      & 0,8                        & 1                        & 0,88                              \\ \hline
\multicolumn{3}{|l|}{Итог:}                                                                                      & 5,38                             \\ \hline
\end{tabular}
\end{center}
\end{table}

Время доступа к нужным данным сократилось, из-за использования новой разработки, следовательно получаем, что ожидаемое время работы тоже сократилось, можно заметить, что почти в половину раз меньше, чем было до внедрения разработки.

Рассчитаем экономическую эффективность разработки, для этого рассчитаем сначала затраты на различные выплаты сотрудникам. 
Расходы на различные виды выплат сотрудникам рассчитываются по формуле (\ref{eq:Zp}):
\begin{equation}\label{eq:Zp}
Z = \sum_{i}^{}{\left(n_{i}\tilde{z_{i}}\left(1+\frac{a_{c}}{100} \right) \right)}
\end{equation}

$Z$ -- суммарный расход на выплаты,

$n_{i}$ -- численность персонала,

$\tilde{z_{i}}$ -- среднегодовая заработная плата работника,

$a_{c}$ -- процент отчислений на социальное страхование, пенсионный фонд и фонд стабилизации.

В объекте задействованы следующие сотрудники:
\begin{itemize}
    \item Ответственный за хранение данных: количество 1; среднегодовая заработная плата работника 360000; процент отчислений 72000,
    \item Ответственный за передачу данных на объект: количество 1; среднегодовая заработная плата работника 400000; процент отчислений 80000.
\end{itemize}
Совершим расчет по приведённым данным:\\
%\begin{flushleft}
%\begin{math}
$Z = \sum_{i}^{}{\left(n_{i}\tilde{z_{i}}\left(1+\frac{a_{c}}{100} \right) \right)} = \\ = \left( 1\cdot 360000 \left(1+\frac{72000}{100} \right)\right) + \left( 1\cdot 400000 \left(1+\frac{80000}{100} \right)\right) = \\ = 320400000$(рублей)
%\end{math}
%\end{flushleft} 

Теперь рассчитаем  стоимость разработки, которая вычисляется по формуле(\ref{eq:stoi}):
\begin{equation}\label{eq:stoi}
S_{p} = Z + P
\end{equation}

$S_{p}$ -- стоимость разработки,

$P$ -- прибыль.

Прибыль составит 42\% от выплат работникам. Рассчитаем стоимость разработки:
\begin{flushleft}
\begin{math}
P = 0,42 \cdot Z = 134570000\end{math} (рублей)\end{flushleft}
\begin{flushleft}
\begin{math}
S_{p} = Z + P = 320400000 + 134570000 = 454970000
\end{math} (рублей)\end{flushleft}

Далее рассчитаем рентабельность затрат на проект, воспользовавшись формулой \ref{eq:rent}:
\begin{equation}\label{eq:rent}
R=\frac{P-P \cdot 0,2}{Z} \cdot 100\%
\end{equation}

$R$ -- рентабельность затрат.

Рассчитываем рентабельность затрат:
\begin{flushleft}
\begin{math}
R=\frac{P-P \cdot 0,2}{Z} \cdot 100\% = \frac{134570000 - 134570000 \cdot 0,2}{320400000} \cdot 100\% = 34\% 
\end{math}\end{flushleft}

Наконец по рассчитанным данным можем показать результат экономического эффекта от усовершенствования объекта. Примем  за $E_{n}$ (норма рентабельности) 25\% . Должно соблюдаться условие, что $R>E_{n}$, получаем $34\%>25\%$ из этого следует, что усовершенствование экономически эффективно.

Рассчитаем срок окупаемости по формуле \ref{eq:stoi}:
\begin{equation}\label{eq:stoi}
T_{per}=\frac{1}{R}
\end{equation}

$T_{per}$ -- срок окупаемости.

Рассчитываем значения срока окупаемости:
\begin{flushleft}
\begin{math}
T_{per}=\frac{1}{R} = \frac{1}{0,34} = 2,9 
\end{math} (лет)\end{flushleft}

Следовательно приблизительное время окупаемости:
\begin{flushleft}
\begin{math}
2,9 \cdot 12 = 108
\end{math} (месяцев)\end{flushleft}  

\pagebreak