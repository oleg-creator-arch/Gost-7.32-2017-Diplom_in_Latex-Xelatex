\section{Реализация информационной системы}
\subsection{Выбор средств реализации}

Для реализации воспользуемся языком программирования $C\#$, этот язык позволит использовать парадигму объектно-ориентированного программирования, с технологией ASP.NET, что позволит использовать архитектуру MVC(Model View Controller), JavaScript-фреймворком для создания пользовательского интерфейса Vue.js. Реализацию будем проводить в среде разработки Visual Studio 2019 Community. Что бы хранить информацию нужна база данных, для реализации такой возможности используем реляционную СУБД MS SQL Server

\subsection{разработка информационной системы}

Покажем модульность структуры программы на рисунке \ref{pic:mod}.

\addimghere{resources/mod}{1}{Модульность структуры программы}{pic:mod}


Разработано три модуля, серверная часть, платформа и панель администратора. Платформа и панель администратора обмениваются данными через серверную часть. Назовем информационную систему, собственным именем, которая было бы связанно с астрономией, есть самая яркая звезда в созвездии орла <<Альтаир>>, так вот назовем разрабатываемую систему в честь этой звезды <<$AltaiR$>>. Используя структуру MVC создадим контролеры, для такого что бы обрабатывать запросы. Контролер для пользователя, включает получение информации о пользователе, её обновление и смена пароля и аватара, для авторизации это сама авторизация и регистрация нового пользователя, для блогов это получение списка доступных блогов и основной информации о них, их редактирование, и приглашение соавтора, для постов это получение списка постов, их редактирование, сохранение или публикация. Покажем на рисунке \ref{pic:get} пример реализации контролера

\addimghere{resources/get}{1}{Пример запроса через контролер}{pic:get}

Данный запрос передает идентификатор блога, после чего контролер должен выдать список опубликованных постов, или выдать ошибку 404.

Для работы с базой данных используется Entity Framework.

C использованием Vue.js реализуем два других модуля(платформа и панель администратора). Платформа отображает по ссылки блог и посты в них, а панель администратора позволит создавать и изменять блоги и посты в них, приглашать и работать в команде с другими авторами. 

\pagebreak
\subsection{Описание пользовательского интерфейса}

Вначале нас встречает главная страница информационной системы <<$AltaiR$>>, с не большой демонстрацией возможностей, предложением зарегистрироваться, и перейти на панель блогов, покажем её на рисунке \ref{pic:start}

\addimghere{resources/start}{1}{Стартовая страница}{pic:start}

перейдя по ссылке, которая формируется таким образом \\<<[адрес сайта]/@[имя блога]>>,

\pagebreak