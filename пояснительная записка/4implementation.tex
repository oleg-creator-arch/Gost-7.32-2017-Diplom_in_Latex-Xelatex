\section{Реализация информационной системы}
\subsection{Выбор средств реализации}

Для реализации воспользуемся языком программирования $C\#$, этот язык позволит использовать парадигму объектно-ориентированного программирования, с технологией ASP.NET, что позволит использовать архитектуру MVC(Model View Controller), JavaScript-фреймворком для создания пользовательского интерфейса Vue.js. Реализацию будем проводить в среде разработки Visual Studio 2019 Community. Что бы хранить информацию нужна база данных, для реализации такой возможности используем реляционную СУБД MS SQL Server. 

Для таких программных средств следует выбрать, персональный компьютер с операционной системой windows 10, не ниже таких минимальных характеристик:
\begin{itemize}
	\item жесткий диск на 80 ГБ, 
	\item видеокарта GeForce GT 310,
	\item блок питания System Power 9 300W,
	\item материнская плата GIGABYTE B450M S2H,
	\item процессор Ryzen 3 1200,
	\item оперативная память DDR4 Huawei 06200244 8Gb RDIMM ECC Reg.
\end{itemize}

\pagebreak

\subsection{Разработка информационной системы}

Покажем модульность структуры программы на рис. \ref{pic:mod}.

\addimghere{resources/mod}{1}{Модульность структуры программы}{pic:mod}


Разработано три модуля, серверная часть, платформа и панель администратора. Платформа и панель администратора обмениваются данными через серверную часть. Назовем информационную систему, собственным именем, которая было бы связанно с астрономией, есть самая яркая звезда в созвездии орла <<Альтаир>>, так вот назовем разрабатываемую систему в честь этой звезды <<$AltaiR$>>. Используя структуру MVC создадим контролеры, для такого что бы обрабатывать запросы. Контролер для пользователя, включает получение информации о пользователе, её обновление и смена пароля с аватаркой, авторизации включает саму авторизацию и регистрацию нового пользователя, для блогов это получение списка доступных блогов и основной информации о них, их редактирование, и приглашение соавтора, для постов это получение списка постов, их редактирование, сохранение или публикация. Покажем на рис. \ref{pic:get} пример реализации контролера.

\addimghere{resources/get}{1}{Пример запроса через контролер}{pic:get}

Данный запрос передает идентификатор блога, после чего контролер должен выдать список опубликованных постов, или выдать ошибку 404.

Для работы с базой данных используется Entity Framework.

C использованием Vue.js реализуем два других модуля(платформа и панель администратора). Платформа отображает по ссылки блог и посты в них, а панель администратора позволит создавать и изменять блоги и посты в них, приглашать и работать в команде с другими авторами. 


\subsection{Описание пользовательского интерфейса}

Вначале нас встречает главная страница информационной системы <<$AltaiR$>>, с не большой демонстрацией возможностей, предложением зарегистрироваться, и перейти на панель блогов, покажем её на рис. \ref{pic:start}.

\addimghere{resources/start}{1}{Стартовая страница}{pic:start}

\pagebreak
Перейдя по ссылке, которая формируется как <<.../@[имя блога]>>, отобразиться список постов в блоге, покажем это на рис. \ref{pic:post}.

\addimghere{resources/post}{1}{Описание блога}{pic:post}


Перейдя на <<ABOUT>> отобразиться информация о блоге, покажем это на рис. \ref{pic:about}.

\addimghere{resources/about}{1}{Список постов}{pic:about}

\pagebreak
При нажатии у поста на <<«READ MORE>>, откроется его контент, покажем это на рис. \ref{pic:contpost}.

\addimghere{resources/contpost}{1}{Пост}{pic:contpost}


Если перейти по кнопки <<BLOG PANEL>>, то мы перейдем с платформы на другой модуль панель администратора, в начале нас встречает окно авторизации, покажем это на рис. \ref{pic:avto}.

\addimghere{resources/avto}{1}{Окно авторизации}{pic:avto}

Если всё введено корректно пользователь станет авторизованным.

\pagebreak
В случаи когда нет аккаунта, надо перейти в <<SIGN-UP>>, и зарегистрироваться, покажем это на рис. \ref{pic:reg}.

\addimghere{resources/reg}{1}{Окно регистрации}{pic:reg}

\pagebreak
После входа будет главная страница панели администратора покажем её на рис. \ref{pic:adm}.

\addimghere{resources/adm}{1}{Панель администратора}{pic:adm}

По кнопки <<CREATE>> создадим новый блог, покажем на рис. \ref{pic:creat} окно создание блога.

\addimghere{resources/creat}{1}{Панель создания блога}{pic:creat}

Если перейти по имени справа сверху откроется панель выбора, покажем её на рис. \ref{pic:vub}, если выбрать <<Logout>> выйдем из профиля, а если <<Profile>>, то перейдем в панель профиля.

\addimghere{resources/vub}{0.25}{Панель выхода из аккаунта или открытие панели профиля}{pic:vub}

\pagebreak
Покажем панель профиля на рис. \ref{pic:prof}.

\addimghere{resources/prof}{1}{Панель профиля}{pic:prof}

В профиле можно выбрать аватар, если нажать <<CHANGE>>, покажем это на рис. \ref{pic:avatar}.

\addimghere{resources/avatar}{1}{Выбор аватара}{pic:avatar}

В панели администратора отображаются список блогов и уведомление о согласии стать соавтором блога, после ответа на уведомление, результат придет к пользователю отправившего запрос, покажем это на рис. \ref{pic:ivent}.

\addimghere{resources/ivent}{1}{Уведомление о соавторстве, и список блогов}{pic:ivent}

Покажем ответ на приглашение на рис. \ref{pic:rezivent}.

\addimghere{resources/rezivent}{1}{Ответ на уведомление о соавторстве}{pic:rezivent}


Если выбрать блог, то отобразиться под панель выбора для работы с постами, приглашение авторов, и получения информации о блоге, покажем её на рис. \ref{pic:raspan}.

\addimghere{resources/raspan}{0.2}{Панель работы с блогам}{pic:raspan}

Если перейти в <<Authors>> появиться и нажать <<INVITE>> откроется окно куда надо будет ввести почту пользователя которого хотим пригласить, покажем это на рис. \ref{pic:inviteavtor}.

\addimghere{resources/inviteavtor}{1}{Приглашение соавтора}{pic:inviteavtor}

Если перейти в <<Blog info>> отобразиться информация о блоге, с возможностью её редактирования, покажем это на рис. \ref{pic:bbinfo}.

\addimghere{resources/bbinfo}{1}{Редактирование информации о блоге}{pic:bbinfo}

Если перейти в <<Posts>>, то отобразиться список постов, и если нажать на <<CREATE>> то создаться пост, покажем это на рис. \ref{pic:cretpost}.

\addimghere{resources/cretpost}{1}{Список постов и создание поста}{pic:cretpost}

\pagebreak
Используется удобный текстовый редактор для редактирования постов, покажем это на рис. \ref{pic:redapost}.

\addimghere{resources/redapost}{1}{Редактирование поста}{pic:redapost}

Есть возможность сохранить кнопка <<SAVE>>, после чего пост сохраниться, но не опубликуется, или <<SAVE\&PUBLISH>> сразу сохранить и опубликовать, и кнопка <<CANCEL>>, которая отменить текущие изменения, покажем это на рис. \ref{pic:redasave}.

\addimghere{resources/redasave}{1}{Публикация поста}{pic:redasave}

\subsection{Вывод}
В результате разработки была разработана информационная система <<$AltaiR$>>, удовлетворяющая всем поставленным критериям. Были выбраны технические и программные средства разработки. Описание как реализовывалась ИС. Показаны, как функционируют модули, что реализован весь возложенный функционал на систему, и инструкция для пользователя.

\pagebreak